\documentclass[12pt]{article}

\usepackage{enumitem}
\usepackage[right=20mm, left=20mm]{geometry}
\usepackage{type1cm}
\usepackage{amssymb}
\usepackage[fleqn]{amsmath}
\usepackage{tikz}
\usepackage{multicol}
\usepackage{makecell}
\setlength{\columnsep}{1pt}
\usepackage{pgfplots}
\usepackage{float}
\usepackage{caption}
\usepackage{subcaption}
% \usepackage{subfig}
\usepackage{graphicx}

\usepackage{indentfirst}
\usepackage{lastpage}  
\usepackage{fancyhdr}
\pagestyle{fancy}

\usepackage{pgfgantt}
\usepackage[unicode=true,pdfusetitle,
 bookmarks=true,bookmarksnumbered=false,bookmarksopen=false,
 breaklinks=false,pdfborder={0 0 1},backref=false,colorlinks=false]
 {hyperref}

\makeatletter
\newenvironment{myalign*}{\ifvmode\else\hfil\null\linebreak\fi
  \hspace*{-\leftmargin}\minipage\textwidth
  \setlength{\abovedisplayskip}{0pt}%
  \setlength{\abovedisplayshortskip}{\abovedisplayskip}%
  \start@align\@ne\st@rredtrue\m@ne}%
{\endalign\endminipage\linebreak}

% Paper size
\topmargin -10mm
\textwidth 170mm
% \oddsidemargin -5mm
% \evensidemargin -5mm
\textheight 220mm

% Font setting
\usepackage{xeCJK}
% \setCJKmainfont{Noto Sans TC}
\setCJKmainfont{kaiu.ttf}


\renewcommand{\footnotesize}{\normalsize} 
\renewcommand{\headrulewidth}{0pt}
\renewcommand{\footrulewidth}{0pt}

\lhead{}
\chead{2023年全國大專校院智慧創新暨跨域整合創作企劃書}
\rhead{}

\lfoot{}
\cfoot{}
\rfoot{ 共 \pageref{LastPage} 頁 第  \thepage   頁} 

\makeatletter
\begin{document}
% \fontsize{14pt}{18pt}\selectfont
% \author{}
\date{}
\usetikzlibrary{automata, positioning, arrows, shapes, fit}
% \maketitle
\tikzset{every state, accepting/.style={double distance=2pt}}
\tikzstyle{block} = [rectangle, draw, fill=white, 
    text centered, rounded corners, minimum height=2em]
\tikzstyle{container} = [rectangle, draw, dashed, inner sep=2em]
\tikzstyle{line} = [draw, -latex']
\captionsetup[figure]{labelfont={bf},name={圖},labelsep=period}
\setlist[itemize]{itemsep=0pt,topsep=0pt,parsep=0pt}
\setlist[enumerate]{itemsep=0pt,topsep=4pt,parsep=0pt}
\setlength{\parindent}{2em}

\noindent
\textbf{參賽隊名:} TraTracker \\
\textbf{作品名稱:} 旅遊紀錄整理工具 \\
\textbf{競賽主題:} 體感互動科技組

\section{創作主題}

\subsection{題目:Travel Tracker - AI對話旅遊紀錄整理工具}

\subsection{實用功能描述}

在旅程途中我們常會使用手機來記錄每趟旅行的點點滴滴,用照片、影片或是文字等載體保留旅程途中的美好回憶。但在旅遊中不會想花太多時間在記錄上,而且旅遊結束的資料量都很可觀,人工整理會耗費相當多的時間。

因此,Travel Tracker針對旅遊記錄、整理的不便,提供使用者便捷的解決方案。使用者可以透過本APP快速記錄軌跡、影像與文字等旅遊資料,我們也會自動從手機載入影像,並在根據影像的拍攝時間,搭配軌跡資訊推算拍攝地點,並提供以拍攝地點分組旅遊資料的整理方式,讓使用者能更快速地找到旅遊途中的美好記錄。

此外,Travel Tracker也提供AI辨識的方式標記圖片類別,讓使用者能快速找到相對應的圖片。最後,Travel Tracker更結合AI對話技術,讓使用者可以輕鬆透過口語與AI對話,快速地整理旅遊中的資料。

\subsection{作品與市場相關產品差異}

本團隊開發之 Travel Tracker 與市面上相關旅遊記錄工具主要有以下差異:

\begin{enumerate}
    \item 其他旅遊記錄工具在整理、編輯或查找資料時主要以手動的方式,當很多資料時會很繁雜。而 Travel Tracker 會透過 AI 初步標記圖像,並以與 AI 對話的方式查找、編輯旅遊資料,讓使用者能輕鬆地整理旅遊資料。
    \item 市面上的旅遊記錄工具大多只能單純記錄影像,並按時間軸條列式呈現旅遊資料。而Travel Tracker則是利用拍攝地點來分組並整理旅遊資料,讓使用者能更快速地依照旅遊地點找到想要的回憶。
    \item 其他旅遊記錄工具很少會同時支援照片、影片、錄音、文字等各式種類的資料,像是山林日誌、健行筆記、Gaia等旅遊記錄APP在記錄中只有相片記錄功能,缺乏影片記錄方式,而Google map+相簿在記錄旅遊資料時,只能在電腦版上編輯文字,無法在APP上新增。因此,本APP將同時支援使用者以相片、影片、文字、錄音的方式來記錄旅遊中的美好。
\end{enumerate}

\section{創意構想}

\subsection{理論基礎}

旅途中應該要好好享受旅程,而不是花時間在新增、整理資料上
照片很多很雜,同個地點拍的照片容易有關連,可以分組
常常會想找某個資料找很久,透過AI協助在照片海中找尋資料

\subsection{設計創新說明}

本工具創新地使用地圖作為照片整理的主要界面,並通過AI技術分析照片內容進行自動標記。除此之外,強調直觀且易於操作的設計也是其創新之處

\subsection{特殊功能描述}

特殊功能包括AI自動辨識並標記照片、地圖上的軌跡顯示、時間軸方式顯示旅遊資料、批量處理旅遊資料等。這些功能共同使Travel Tracker成為一個強大的旅遊紀錄整理工具。

\section{系統架構}

\subsection{架構說明}

Travel Tracker採用多層架構,包括用戶界面層、業務邏輯層、資料存取層和資料庫層。每一層都有其特定的功能和責任,共同支持整個系統的運作。

\subsection{「人機介面設計」(UI)與「使用者體驗」(UX)設計}


Travel Tracker的UI設計強調簡潔和直觀,以使操作更為快速方便。UX則著重於用戶的整體體驗,包括美觀的視覺設計、流暢的互動和符合使用者需求的功能。整體設計旨在提供一個愉悅和高效的使用體驗。

\section{計劃管理}

% \begin{table}[H]      
%   \centering
%   \begin{tabular}{|c|c|p{13cm}|}
%     \hline
%     工作階段 & 工作日數 & 工作內容 \\ \hline
%     第一階段 & 14日 & 創意發想、需求分析與規劃階段:\newline  
%       1. 用戶需求收集及分析、功能需求深度討論、市場分析。\newline
%       2. 根據需求選定技術、設計資料結構及資料庫模型。\newline
%       3. 開發計劃與時間表的制定及製作專案介紹。\\ \hline
%     第二階段 & 14日 & 設計階段:\newline
%       1. 繪製系統架構圖,撰寫技術設計文檔。\newline
%       2. 設計前後端介面。\newline
%       3. 研究直播系統的工作原理、Judge 系統的實現方式。\\ \hline
%     第三階段 & 21日 & 系統開發與文件準備階段:\newline
%       1. 前後端開發:\newline
%       - 前端:實現教師的章節編輯頁及教學頁面的互動式講義等。\newline
%       - 後端:開發API,完成基礎CRUD、websocket,部屬docker。\newline
%       2. 文件與影片製作\newline
%       3. 整理產品的用戶故事和場景。 \\ \hline    
%     第四階段 & 21日 & 功能開發與初步測試階段:\newline
%      1. 前後端功能開發:\newline
%      - 前端:完成互動式講義編輯頁面、教學頁面的工具欄等主要功能。\newline
%      - 後端:完成直播系統、Judge系統的開發,並將系統部署到服務器。\newline
%      2. 初步功能測試及撰寫開發者文件 \\ \hline
%     第五階段 & 21日 & 使用者測試與優化階段:\newline
%      1. 進行使用者測試訪談並收集反饋,了解使用過程中遇到的問題。\newline
%      2. 根據收集的數據與反饋進行分析,找出需要改進的部分。\newline
%      3. 根據分析結果對系統進行優化和調整。 \\ \hline
%     第六階段 & 21日 & 最終測試與優化階段:\newline
%      1. 進行完整系統測試,包括功能、性能、安全等方面。\newline
%      2. 根據測試結果進行系統性能優化。\newline
%      3. 撰寫與記錄測試過程和結果的設計測試文件。 \\ \hline
%     第七階段 & 8日  & 準備決賽階段:\newline
%      1. 撰寫系統使用手冊。\newline
%      2. 準備決賽所需的各種材料和準備工作。 \\ \hline
%   \end{tabular}
%   \caption{計劃管理}
% \end{table}

% \begin{figure}[H]
%   \centering
%   \begin{ganttchart}[
%     y unit title=0.6cm,
%     y unit chart=0.7cm,
%     x unit=0.7cm,
%     vgrid,hgrid, 
%     title height=1,
%     progress label text={},
%     bar height=0.8,
%     bar top shift=0.1,
%     ]{1}{18} % 按照您的計劃,我們有約 18 週的時間。
%     %labels
%     \gantttitlelist{1,...,18}{1} \\ % 生成 1 到 18 的標籤
    
%     %tasks
%     \ganttbar{工作階段 1}{1}{2} \\
%     \ganttbar{工作階段 2}{3}{4} \\
%     \ganttbar{工作階段 3}{5}{7} \\
%     \ganttbar{工作階段 4}{8}{10} \\
%     \ganttbar{工作階段 5}{11}{13} \\
%     \ganttbar{工作階段 6}{14}{16} \\
%     \ganttbar{工作階段 7}{17}{18}
%   \end{ganttchart}
%   \caption{工作階段的甘特圖}  
% \end{figure}

\section{修改舊作參賽說明}
  本專案開發之作品未使用團隊成員曾獲競賽獎勵之作品。
\section{軟體清單}
\begin{itemize}
  \item 作業系統環境:Windows、MacOS
  \item 主要開發程式語言:Dart
  \item 專案支援語言:中文
  \item 開發環境:
  \begin{itemize}
    \item Android
    \item IOS
    \item Visual Studio Code
    \item Flutter
    \item Git, Github
  \end{itemize}
  \item 專案成果預定授權條款:本專案開發產品授權條款使用 BSD 3-Clause 宣告。
\end{itemize}

\section{權力分配}
依著作權法第 40 條之規定,由參賽學生與指導教授均等共有。

\end{document}