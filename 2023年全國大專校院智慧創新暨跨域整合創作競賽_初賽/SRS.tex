\documentclass[12pt]{article}

\usepackage{enumitem}
\usepackage[right=20mm, left=20mm]{geometry}
\usepackage{type1cm}
\usepackage{amssymb}
\usepackage[fleqn]{amsmath}
\usepackage{tikz}
\usepackage{multicol}
\usepackage{makecell}
\setlength{\columnsep}{1pt}
\usepackage{pgfplots}
\usepackage{float}
\usepackage{caption}
\usepackage{subcaption}
% \usepackage{subfig}
\usepackage{graphicx}
\usepackage{array}

\usepackage{indentfirst}
\usepackage{lastpage}  
\usepackage{fancyhdr}

% long table
\usepackage{longtable}
\pagestyle{fancy}

\usepackage[unicode=true,pdfusetitle,
 bookmarks=true,bookmarksnumbered=false,bookmarksopen=false,
 breaklinks=false,pdfborder={0 0 1},backref=false,colorlinks=false]
 {hyperref}

\makeatletter
\newenvironment{myalign*}{\ifvmode\else\hfil\null\linebreak\fi
  \hspace*{-\leftmargin}\minipage\textwidth
  \setlength{\abovedisplayskip}{0pt}%
  \setlength{\abovedisplayshortskip}{\abovedisplayskip}%
  \start@align\@ne\st@rredtrue\m@ne}%
{\endalign\endminipage\linebreak}

% Paper size
\topmargin -10mm
\textwidth 170mm
% \oddsidemargin -5mm
% \evensidemargin -5mm
\textheight 220mm

% Font setting
\usepackage{xeCJK}
% \setCJKmainfont{Noto Sans TC}
\setCJKmainfont{kaiu.ttf}


\renewcommand{\footnotesize}{\normalsize} 
\renewcommand{\headrulewidth}{0pt}
\renewcommand{\footrulewidth}{0pt}

\lhead{}
\chead{2023年全國大專校院智慧創新暨跨域整合創作競賽系統需求書}
\rhead{}

\lfoot{}
\cfoot{}
\rfoot{ 共 \pageref{LastPage} 頁 第  \thepage   頁} 

\makeatletter
\begin{document}
% \fontsize{14pt}{18pt}\selectfont
% \author{}
\date{}
\usetikzlibrary{automata, positioning, arrows}
% \maketitle
\tikzset{every state, accepting/.style={double distance=2pt}}
\captionsetup[figure]{labelfont={bf},name={圖},labelsep=period}
\renewcommand{\arraystretch}{1.45}

\noindent
\textbf{參賽隊名:} TraTracker \\
\textbf{作品名稱:} 旅遊紀錄整理工具 \\
\textbf{系統名稱:} Travel Tracker

\section{系統目的與範圍}

Travel Tracker 的主要目的是解決旅遊記錄、整理的不便,創造一個便捷的方式在旅程中輕鬆地記錄和整理旅遊資料。通過使用地圖、AI對話和自動標籤等技術,讓使用者能快速找到和整理旅遊途中的美好記錄。

\section{Travel Tracker 提供的服務}
\begin{enumerate}
  \item \textbf{快速記錄軌跡與多媒體資料}:不用花時間開啟各種程式,能以最快的速度記錄回憶。
  \item \textbf{自動影像載入與地點推算}:自動從手機載入影像,並根據拍攝時間與軌跡資訊推算拍攝地點。
  \item \textbf{以拍攝地點分組旅遊資料}:利用地理資訊,讓使用者依照地點快速找到旅遊回憶。
  \item \textbf{AI辨識標記與對話整理}:使用AI技術自動標記圖像類別,並透過AI對話技術輕鬆整理旅遊資料。
  \item \textbf{全方位多媒體支援}:同時支援照片、影片、錄音、文字等資料形式,提供多元化的記錄方式。
\end{enumerate}

\section{Travel Tracker 涵蓋的技術}
\begin{enumerate}
  \item \textbf{地圖軌跡整合技術}:使用地圖作為主要界面,結合GPS訊號和拍攝地點的推算,使旅遊資料更具關聯性和直觀性。
  \item \textbf{人工智慧辨識與對話技術}:利用AI分析照片內容進行自動標記,並結合自然語言處理技術,讓使用者能夠與AI進行對話,快速整理旅遊資料。
  \item \textbf{跨平台兼容性}:支援不同的移動設備,透過適應性設計讓使用者在不同設備上都能夠方便地使用。
\end{enumerate}

\section{系統非功能需求}

\begin{longtable}{|c|p{5.2cm}|p{7.5cm}|}
  \hline
  \textbf{非功能需求編號} & \textbf{功能名稱} & \textbf{功能需求描述} \\
  \hline
  \endfirsthead
  \hline
  \textbf{非功能需求編號} & \textbf{功能名稱} & \textbf{功能需求描述} \\
  \hline
  \endhead
  TT-NF-01 & 快速記錄 & 桌面小工具要取得快速、操作簡單明瞭方便 \\
  \hline
  TT-NF-02 & 大型資料處理效率 & 系統要能處理大型媒體檔案不會當機的效能 \\
  \hline
  TT-NF-03 & 使用者界面 & 系統界面必須直覺易用,方便使用者操作和理解。視覺設計需統一並符合現代化的美學標準。 \\
  \hline
  TT-NF-04 & 可用性 & 系統應具備高可用性,保證穩定的運行。所有功能都應在普通網絡條件下迅速響應。 \\
  \hline
  TT-NF-05 & 安全性 & 所有個人和旅遊相關資料必須安全存儲,並符合相關隱私法規。應用程式應具有適當的權限管理,防止未經授權的訪問。 \\
  \hline
  TT-NF-06 & 兼容性 & 應用程式應支援主流的移動操作系統和版本。地圖和多媒體資料應在不同的裝置和解析度上正常顯示。 \\
  \hline
  TT-NF-07 & 地圖視圖控制 & 系統應支援不同地圖主題和樣式的切換。定位功能必須精確且迅速反應。 \\
  \hline
  TT-NF-08 & 資料匯入、匯出 & 資料匯入和匯出應支援通用格式,如CSV、GPX等。匯入功能應能自動識別和配對相關媒體和軌跡。 \\
  \hline
  TT-NF-09 & 旅程管理 & 旅程切換和管理應流暢且直覺,支援多旅程同時記錄和顯示。提供完整的旅程統計資訊,如總距離、時間等。 \\
  \hline
  TT-NF-10 & 桌面小工具 & 桌面小工具需提供快速記錄和查看功能,並且操作簡單明瞭方便。 \\
  \hline
  TT-NF-11 & AI 功能 & AI對話和自動辨識功能應準確並能有效協助使用者進行篩選、編輯等操作。 \\
  \hline
\end{longtable}

\section{系統功能需求}

\begin{itemize}
  \item 旅程:包含一次旅遊的所有軌跡、旅遊資料
  \item 軌跡:使用者用GPS記錄的路徑
  \item 旅遊資料:照片、影片、錄音、文字等多媒體
\end{itemize}

\subsection{記錄功能 RC}

\begin{longtable}{|c|p{4.3cm}|p{8.9cm}|}
  \hline
  \textbf{功能需求編號} & \textbf{功能名稱} & \textbf{功能需求描述} \\
  \hline
  \endfirsthead
  \hline
  \textbf{功能需求編號} & \textbf{功能名稱} & \textbf{功能需求描述} \\
  \hline
  \endhead
  TT-F-RC-01 & 記錄軌跡 & 使用者可以通過GPS記錄旅程軌跡 \\
  \hline
  TT-F-RC-02 & 暫停記錄軌跡 & 允許使用者暫停記錄旅程軌跡 \\
  \hline
  TT-F-RC-03 & 從APP開啟相機 & 使用者可以直接從應用程式開啟相機進行拍照 \\
  \hline
  TT-F-RC-04 & 從APP錄音 & 使用者可以直接從應用程式開啟麥克風進行錄音 \\
  \hline
  TT-F-RC-05 & 記錄文字 & 允許使用者記錄文字內容,如日記或心得 \\
  \hline
  TT-F-RC-06 & 快速記錄 & 使用者可以通過桌面小工具或通知列快速記錄文字、錄音等等 \\
  \hline
  TT-F-RC-07 & 自動抓取媒體資料 & 自動從手機的相簿或其他媒體資料夾中抓取相關旅遊媒體資料 \\
  \hline
\end{longtable}


\subsection{檢視軌跡與旅遊資料 RV}

\begin{longtable}{|c|p{4.3cm}|p{8.9cm}|}
  \hline
  \textbf{功能需求編號} & \textbf{功能名稱} & \textbf{功能需求描述} \\
  \hline
  \endfirsthead
  \hline
  \textbf{功能需求編號} & \textbf{功能名稱} & \textbf{功能需求描述} \\
  \hline
  \endhead
  TT-F-RV-01 & 地圖上顯示軌跡 & 在地圖上展示使用者的旅程軌跡 \\
  \hline
  TT-F-RV-02 & 軌跡上顯示旅遊資料 & 軌跡上顯示旅遊相關資料,如照片、影片、文字等 \\
  \hline
  TT-F-RV-03 & 地圖旁以時間軸方式顯示旅遊資料 & 旅遊資料以時間軸形式展示,方便查看 \\
  \hline
  TT-F-RV-04 & 格狀顯示旅遊資料 & 使用者能切換到格狀頁面檢視旅遊資料 \\
  \hline
  TT-F-RV-05 & 條列式詳細顯示旅遊資料 & 使用者能切換到條列式頁面,以便於檢視旅遊資料的詳細內容 \\
  \hline
  TT-F-RV-06 & 檢視個別旅遊資料 & 查看單一旅遊資料的詳細資訊 \\
  \hline
  TT-F-RV-07 & 利用標籤、日期等項目篩選旅遊資料 & 根據日期、標籤等條件篩選和搜尋旅遊資料 \\
  \hline
  TT-F-RV-08 & 用地理位置篩選旅遊資料 & 一次選取某個位置周圍的旅遊資料 \\
  \hline
  TT-F-RV-09 & 與AI對話來篩選旅遊資料 & 通過AI對話界面協助篩選旅遊資料 \\
  \hline
\end{longtable}

\subsection{編輯軌跡與旅遊資料 WR}

\begin{longtable}{|c|p{4.3cm}|p{8.9cm}|}
  \hline
  \textbf{功能需求編號} & \textbf{功能名稱} & \textbf{功能需求描述} \\
  \hline
  \endfirsthead
  \hline
  \textbf{功能需求編號} & \textbf{功能名稱} & \textbf{功能需求描述} \\
  \hline
  \endhead
  TT-F-WR-01 & 軌跡平滑化 & 將記錄的軌跡進行平滑處理,使之更符合實際路線 \\
  \hline
  TT-F-WR-02 & 拉伸軌跡 & 允許使用者手動調整軌跡的形狀 \\
  \hline
  TT-F-WR-03 & 偵測軌跡停留位置並簡化 & 能自動識別軌跡中的停留位置,並將其 GPS 誤差造成的一團路徑簡化為一個點 \\
  \hline
  TT-F-WR-04 & 編輯旅遊資料的標籤 & 允許使用者為旅遊資料添加或修改標籤 \\
  \hline
  TT-F-WR-05 & 編輯旅遊資料的日期 & 允許使用者為旅遊資料添加或修改日期 \\
  \hline
  TT-F-WR-06 & 刪除旅遊資料 & 允許使用者刪除指定的旅遊資料 \\
  \hline
  TT-F-WR-07 & 批量處理旅遊資料 & 選取多個旅遊資料並一起編輯或刪除 \\
  \hline
  TT-F-WR-08 & 與AI對話來選取並編輯與刪除旅遊資料 & 通過AI對話協助選取、編輯或刪除旅遊資料 \\
  \hline
  TT-F-WR-09 & AI自動辨識並標記照片 & AI會自動分析並標記照片中的物體或景物 \\
  \hline
\end{longtable}


\subsection{地圖視圖控制 MC}

\begin{longtable}{|c|p{4.3cm}|p{8.9cm}|}
  \hline
  \textbf{功能需求編號} & \textbf{功能名稱} & \textbf{功能需求描述} \\
  \hline
  \endfirsthead
  \hline
  \textbf{功能需求編號} & \textbf{功能名稱} & \textbf{功能需求描述} \\
  \hline
  \endhead
  TT-F-MC-01 & 基礎控制動作 & 使用者可以縮放、移動、旋轉地圖 \\
  \hline
  TT-F-MC-02 & 定位至目前位置 & 讓使用者能快速定位到當前位置 \\
  \hline
  TT-F-MC-03 & 定位至指定旅遊資料 & 允許使用者快速定位至特定旅遊資料 \\
  \hline
  TT-F-MC-04 & 定位至指定軌跡 & 允許使用者快速定位至特定軌跡 \\
  \hline
  TT-F-MC-05 & 地圖樣式選擇 & 使用者可以選擇不同的地圖主題和樣式,如魯地圖、OpenStreetMap、夜間模式、衛星圖像等 \\
  \hline
  TT-F-MC-06 & 離線地圖 & 使用者可以下載地圖,以便在沒有網路的情況下使用 \\
  \hline
\end{longtable}

\subsection{資料匯入、匯出 EP}

\begin{longtable}{|c|p{4.3cm}|p{8.9cm}|}
  \hline
  \textbf{功能需求編號} & \textbf{功能名稱} & \textbf{功能需求描述} \\
  \hline
  \endfirsthead
  \hline
  \textbf{功能需求編號} & \textbf{功能名稱} & \textbf{功能需求描述} \\
  \hline
  \endhead
  TT-F-EP-01 & 匯出旅程、軌跡、旅遊資料 & 允許使用者將各式資料匯出為特定格式的檔案 \\
  \hline
  TT-F-EP-02 & 匯入旅程、軌跡、旅遊資料 & 允許使用者將各式資料匯入,並且能根據軌跡的時間戳記從手機內自動讀取相應的媒體 \\
  \hline
\end{longtable}

\subsection{旅程管理 TM}

\begin{longtable}{|c|p{4.3cm}|p{8.9cm}|}
  \hline
  \textbf{功能需求編號} & \textbf{功能名稱} & \textbf{功能需求描述} \\
  \hline
  \endfirsthead
  \hline
  \textbf{功能需求編號} & \textbf{功能名稱} & \textbf{功能需求描述} \\
  \hline
  \endhead
  TT-F-TM-01 & 創建新旅程 & 允許使用者創建新的旅程 \\
  \hline
  TT-F-TM-02 & 切換目前記錄旅程\footnote[1] & 允許使用者切換正在記錄的旅程 \\
  \hline
  TT-F-TM-03 & 切換目前顯示旅程 & 允許使用者選擇目前正在顯示的旅程以方便查看 \\
  \hline
  TT-F-TM-04 & 一鍵回到目前旅程 & 允許使用者快速將當前記錄中的旅程做為顯示中的旅程 \\
  \hline
  TT-F-TM-05 & 旅程重新命名 & 允許使用者對旅程進行重新命名 \\
  \hline
  TT-F-TM-06 & 顯示旅程總距離、時間等等 & 提供使用者查看旅程的總距離和時間等資訊,以便了解整體旅程的概況 \\
  \hline
\end{longtable}

\footnotetext[1]{顯示中與記錄中的旅程是獨立的,為了讓使用者可以在記錄B旅遊時回去看A旅遊的資料}

  
\section{一般性的系統功能操作使用案例之劇本(Scenario)}

\begin{longtable}{|l|p{13.25cm}|}
  \hline
  \textbf{使用者案例} & \textbf{TT-UC-01} \\
  \hline
  \textbf{案例名稱} & 軌跡和資料記錄 \\
  \hline
  \textbf{相關功能性需求} & 記錄軌跡、暫停記錄軌跡、從APP開啟相機、從APP錄音、記錄文字、快速記錄 \\
  \hline
  \textbf{使用者} & 一般使用者 \\
  \hline
  \textbf{前置條件} & 使用者已安裝相應的應用程式並準備記錄旅遊資料 \\
  \hline
  \textbf{說明} & 使用者可以在旅途中使用手機上應用程式記錄軌跡、拍照、錄音或文字,並可以使用快捷功能更快速的記錄資料,也可以隨時暫停記錄。 \\
  \hline
  \textbf{使用者操作} & 
  1. 使用者開啟記錄功能開始記錄軌跡,並有暫停鍵或結束鍵以提供暫停或結束記錄。\newline
  2. 使用者可以從APP中開啟手機上相機功能進行拍照或錄影。\newline
  3. 使用者可以從APP中開啟錄音功能開始錄音。\newline
  4. 使用者可以在APP中開啟文字記錄功能開始記錄文字。\newline
  5. 使用者可透過APP提供的桌面小工具或通知列選項,讓使用者可以不用打開APP就可以直接快速地記錄旅遊資料。 \\
  \hline
  \textbf{例外處理} & 若使用者的裝置沒有授予相關權限(例如鏡頭、錄音),則應提供適當的提示訊息。 \\
  \hline
  \textbf{測試方案} & 測試在沒有權限的情況下開啟相關功能,確認有提供提示訊息。 \\
  \hline
\end{longtable}

\begin{longtable}{|l|p{13.25cm}|}
  \hline
  \textbf{使用者案例} & \textbf{TT-UC-02} \\
  \hline
  \textbf{案例名稱} & 檢視旅遊資料 \\
  \hline
  \textbf{相關功能性需求} & 軌跡上顯示旅遊資料、地圖旁以時間軸方式顯示旅遊資料、格狀顯示旅遊資料、條列式詳細顯示旅遊資料、檢視個別旅遊資料 \\
  \hline
  \textbf{使用者} & 一般使用者 \\
  \hline
  \textbf{前置條件} & 需將旅遊資料記錄好,包含行進軌跡、影像、錄音或記錄之文字。 \\
  \hline
  \textbf{說明} & 使用者可以依自己想瀏覽旅遊資料的形式選擇合適的旅遊資料瀏覽方式,瀏覽方式有:在地圖軌跡上顯示旅遊資料、在地圖旁以時間軸方式顯示對應軌跡的旅遊資料、格狀式顯示旅遊資料或條列式顯示旅遊資料,此外,也可以檢視個別或單一影像、錄音或文字。 \\
  \hline
  \textbf{描述(操作流程)} & 
  1. 使用者可直接在地圖頁籤縮放地圖來查看軌跡上的影像。\newline
  2. 使用者在地圖頁籤,可在地圖旁看到時間軸,以上下滑動的方式瀏覽旅遊資料,並可對應到地圖上的軌跡點。\newline
  3. 使用者可以向左滑,或是點擊資料頁籤,將時間軸上的資料展開檢視。\newline
  4. 使用者在資料頁籤時,從檢視模式選項選擇以格狀方式顯示旅遊資料。\newline
  5. 使用者在資料頁籤時,從檢視模式選項選擇以條列方式顯示旅遊資料。\newline
  6. 使用者可在任何模式點選個別旅遊資料並瀏覽。 \\
  \hline
  \textbf{例外處理} & 若使用者未記錄行進軌跡、影像、錄音或文字,則系統會顯示「無資料顯示」。 \\
  \hline
  \textbf{測試方案} & 使用者在未新增旅遊資料並點選瀏覽紀錄之功能時,確認系統是否顯示「無資料顯示」。 \\
  \hline
\end{longtable}

\begin{longtable}{|l|p{13.25cm}|}
  \hline
  \textbf{使用者案例} & \textbf{TT-UC-03} \\
  \hline
  \textbf{案例名稱} & 軌跡管理 \\
  \hline
  \textbf{相關功能性需求} & 軌跡平滑化、拉伸軌跡、偵測軌跡停留位置並簡化 \\
  \hline
  \textbf{使用者} & 一般使用者 \\
  \hline
  \textbf{前置條件} & 使用者已記錄完軌跡。 \\
  \hline
  \textbf{說明} & 使用者可以對已經記錄的軌跡進行平滑化、拉伸或是偵測軌跡停留位置並簡化。 \\
  \hline
  \textbf{描述(操作流程)} & 
  1. 使用者選擇已記錄的軌跡。\newline
  2. 使用者選擇平滑化軌跡。\newline
  3. 使用者選擇拉伸軌跡。\newline
  4. 若平滑化軌跡後軌跡有雜亂的點,使用者可以選擇偵測軌跡停留位置並將雜亂的軌跡路線簡化成一個點。 \\
  \hline
  \textbf{例外處理} & 使用者沒有記錄軌跡,則系統應提示「無軌跡」。 \\
  \hline
  \textbf{測試方案} & 測試不輸入軌跡,確保所有修改功能對使用者顯示對應之提示。 \\
  \hline
\end{longtable}

\begin{longtable}{|l|p{13.25cm}|}
  \hline
  \textbf{使用者案例} & \textbf{TT-UC-04} \\
  \hline
  \textbf{案例名稱} & 旅遊資料管理 \\
  \hline
  \textbf{相關功能性需求} & 編輯旅遊資料的標籤、編輯旅遊資料的日期、刪除旅遊資料、批量處理旅遊資料、與AI對話來選取並編輯與刪除旅遊資料、AI自動辨識並標記照片 \\
  \hline
  \textbf{使用者} & 一般使用者 \\
  \hline
  \textbf{前置條件} & 使用者已記錄旅遊資料(影像、錄音或文字) \\
  \hline
  \textbf{說明} & 使用者可以編輯標籤、日期,刪除資料,或批量處理。此外,還可以透過AI的協助進行更智能的資料管理,例如自動辨識標記照片並在搜尋欄中搜尋已標記的照片。 \\
  \hline
  \textbf{使用者操作} & 
  1. 使用者選擇旅遊資料。\newline
  2. 使用者編輯標籤和日期。\newline
  3. 使用者刪除旅遊資料。\newline
  4. 使用者批量處理旅遊資料。\newline
  5. 使用者與AI對話來選取、編輯或刪除資料。\newline
  6. AI自動辨識並標記照片。\newline
  7. 使用者在搜尋欄中搜尋已標記的照片。 \\
  \hline
  \textbf{例外處理} & 若資料缺失、系統無法正確解析或AI無法辨別標記照片,則應提供適當的錯誤訊息。 \\
  \hline
  \textbf{測試方案} & 
  1. 使用者在未新增旅遊資料時,確認系統是否顯示「無資料」。\newline
  2. 測試將照片放置系統中,AI是否可以自動辨認並標記照片。\newline
  3. 測試在搜尋欄中搜尋已標記的內容,確認系統是否可顯示相對應的內容。 \\
  \hline
\end{longtable}


\end{document}