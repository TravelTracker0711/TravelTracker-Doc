\documentclass[12pt]{article}

% 能使用 itemize、enumerate
\usepackage{enumitem}
\setlist[itemize]{noitemsep, topsep=0pt, parsep=0pt, partopsep=0pt}
\setlist[itemize,2]{label=$\circ$, noitemsep, topsep=0pt, parsep=0pt, partopsep=0pt}
\setlist[enumerate]{noitemsep, topsep=0pt, parsep=0.5em, partopsep=0pt}

% 設定頁面邊界
\usepackage[right=20mm, left=20mm, top=30mm, bottom=20mm]{geometry}
% \usepackage[right=20mm, left=20mm, top=20mm, bottom=20mm, headheight=0pt, headsep=20pt]{geometry}

% 設定 type1 字型,基於 Bezier 曲線
\usepackage{type1cm}

% 數學符號
% \usepackage{amssymb}
% \usepackage[fleqn]{amsmath}

% 繪製圖形
% \usepackage{tikz}

% 多欄排版
\usepackage{multicol}

% 設定區域行距
\usepackage{setspace}

% 設定段落行距
\setlength{\parskip}{1em}

% 都是圖表的東西
% \usepackage{makecell}
% \setlength{\columnsep}{1pt}
% \usepackage{pgfplots}
% % \usepackage{subfig}

% 顯示圖片
\usepackage{float}
\usepackage{graphicx}
\usepackage{caption}
\usepackage{subcaption}
\captionsetup[figure]{name={圖},labelsep=period}


% 章節首段縮排
\usepackage{indentfirst}

% 頁碼
\usepackage{lastpage}
\usepackage{fancyhdr}
\pagestyle{fancy}
\renewcommand{\headrulewidth}{0pt}

\lhead{Travel Tracker - 旅遊紀錄整理工具}
\chead{}
\rhead{鄧暐宣、簡蔚驊、蔡維軒、黃怡瑄}

\lfoot{}
\cfoot{}
\rfoot{ 共 \pageref{LastPage} 頁 第  \thepage   頁} 

% 超連結、目錄跳轉
\usepackage[unicode=true,pdfusetitle,
 bookmarks=true,bookmarksnumbered=false,bookmarksopen=false,
 breaklinks=false,pdfborder={0 0 0},backref=false,colorlinks=false]
 {hyperref}

% 英文字體設定
\usepackage{fontspec}
\setmainfont[
  Path=font/,
  UprightFont={Calibri Regular.ttf},
  BoldFont={Calibri Bold.ttf},
  ItalicFont={Calibri Italic.ttf},
  BoldItalicFont={Calibri Bold Italic.ttf},
  FontFace={l}{n}{Calibri Light.ttf},
  FontFace={l}{it}{Calibri Light Italic.ttf}
]{Calibri}
% 中文字體設定
\usepackage{xeCJK}
% \setCJKmainfont[
%   Path=font/,
%   Extension=.ttc,
%   UprightFont={msjh},
%   BoldFont={msjhbd}
% ]{微軟正黑體}
\setCJKmainfont[Path=font/, UprightFont={kaiu.ttf}]{標楷體}

% 大標題樣式
\usepackage{titling}

\preauthor{\begin{flushright}\normalsize}
\postauthor{\end{flushright}}

% 標題樣式
\usepackage{titlesec}

% \titleformat{\section}{\centering\Large\bfseries}{}{0em}{}[\titlerule]
% \titlespacing*{\section}{0pt}{0em}{1em}
\titlespacing{\section}{0pt}{0em}{0em}

% \titleformat{\subsection}{\centering\large\bfseries}{}{0em}{}[]
% \titlespacing*{\subsection}{0pt}{0em}{0em}
\titlespacing{\subsection}{0pt}{0em}{0em}

\titlespacing*{\subsubsection}{0pt}{0em}{0em}

% 目錄樣式
\usepackage{titletoc}

% \titlecontents{section}[0em]{\setstretch{1.5}\bfseries}{\hspace{0em}}{}{\titlerule*[0.5pc]{.}\contentspage}

% 目錄名稱
\renewcommand{\contentsname}{目錄}

% 字體顏色
\usepackage{xcolor}

% \let\originaltextbf\textbf
% \renewcommand{\textbf}[1]{\originaltextbf{\textcolor{red}{#1}}}
\newcommand{\textbfred}[1]{\textcolor{red}{\textbf{#1}}}

%%%%%%%%%% 文件開始 %%%%%%%%%%
\begin{document}

% tikz 繪圖設定
% \usetikzlibrary{automata, positioning, arrows}
% \tikzset{every state, accepting/.style={double distance=2pt}}

% caption 設定
% \captionsetup[figure]{labelfont={bf},name={圖},labelsep=period}

% 標題
% 設定標題、作者、日期
\title{}
\author{}
\date{}
% \maketitle
\thispagestyle{fancy}

\setstretch{1.4}

% 目錄(TOC)
% \tableofcontents

% \newpage

\section{摘要}

本專題研究旨在開發一款提供使用者快速、便捷方式記錄並整理旅遊回憶的APP。為了解決旅遊中記錄不便和後續整理困難的問題,我們設計的APP能讓使用者在旅行過程中迅速記錄軌跡、影像、和文字等資料,同時透過影像拍攝時間自動推算拍攝地點,以便使用者快速瀏覽和查找旅行記錄。該APP採用了Flutter框架,以實現跨平台的使用,並應用了Clean Architecture和MVVM設計模式以達到架構的清晰劃分和模組的可重用性。

目前的APP原型已經具備基本的用戶介面和旅程記錄、資料管理功能,並在「2023全國大專校院智慧創新暨跨域整合創作競賽」入圍決賽,未來我們將繼續完善與拓展應用功能,期望為廣大使用者帶來更完整的旅行記錄和管理解決方案。

\section{研究動機與目的}

在旅程途中我們常會使用照片、影片或是文字等載體保留旅程途中的美好回憶。但在旅遊中不會想花太多時間在記錄上,而且旅遊結束的資料量都很可觀,人工整理會耗費相當多的時間。

因此,我們想開發一款針對整理旅遊記錄的不便,提供使用者便捷的解決方案的APP。使用者可以快速記錄軌跡、影像與文字等旅遊資料,也能自動從手機載入影像,並根據影像的拍攝時間推算拍攝地點。讓使用者能更快速地找到旅遊途中的美好記錄。

% 此外,Travel Tracker也提供AI辨識的方式標記圖片類別,讓使用者能快速找到相對應的圖片。最後,Travel Tracker更結合AI對話技術,讓使用者可以輕鬆透過口語與AI對話,快速地整理旅遊中的資料。

% 我們推出的TravelTracker主要希望解決旅遊記錄和整理的不便,提供快速的軌跡、影像與文字記錄功能。更重要的,我們結合AI辨識和對話技術,讓使用者能更輕鬆地整理和管理旅行中的美好時刻。

% 我們經過問卷調查與自身經驗,得出了旅遊中時常有以下的問題:

% 1. 旅途中花太多時間整理資料,或是照片太雜亂: 出去玩應該要把最多的時間放在享受過程,最少的時間在記錄與整理上,如果出去玩都在看著手機翻找照片,反而會本末倒置,失去去旅遊、體驗生活的意義。

% 2. 使用旅遊APP常常沒辦法自動匯入資料,需要一個一個加入: 旅遊過程會常常拍照,產生很多資料,如果要手動加入APP的話會花太多時間做重複性的事情,失去寶貴的時間,又沒辦法確認是否已經加入所有相關的照片。

% 3. 沒辦法支援所有想記錄的資料格式,如照片、文字、錄音等等:單純的相簿沒辦法存放文字,大部分日記APP又得自己慢慢插入圖片,沒辦法一次性將所有的資料整合起來,常常要在很多不同的位置找到自己的記錄。

% 4. 複雜的系統操作方式時常不夠直覺:旅遊過程如果需要花很多時間開啟APP,或是找到合適的位置記錄心中的想法,很可能會忘記本來旅途獲得的感受,也會用太多時間在解決系統問題上。
% 5. 簡單的系統提供的功能太陽春:與上一點相對,系統簡單雖然操作不複雜,但也不能支援所有整理的需求,反而會讓整理的過程變慢

\section{研究材料與方法}

\subsection{開發框架}





\subsection{架構設計}

我們參考了 Clean architecture 與 MVVM 來設計架構,引入依賴反轉、雙向數據綁定等特性,旨在減少模組之間的依賴與增加可重用性,使得各部分能獨立開發和測試,並使得未來對系統的修改和擴展更為容易。此外,明確的架構和分層可以幫助團隊成員更快地瞭解專案結構,並減少彼此之間的工作衝突。

我們定義了兩種資料容器:

\begin{enumerate}
    \item Entity:作為系統中的主要資料容器,包含旅程、軌跡、座標點等核心資料。
    \item QueryItem:向外部資料庫(如SQL)請求數據獲得的原始結果,格式與資料表的定義相同。
\end{enumerate}

\begin{figure}[H]
    \centering
    \includegraphics[width=0.8\textwidth]{assets/TT分層依賴圖.png}
    \caption{分層依賴圖}
    \label{分層依賴圖}
\end{figure}

我們將系統架構分為 Use Case、Repository、Data Source、View Model 和 UI 五層,圖~\ref{分層依賴圖} 為分層依賴圖,展示了各層內部的元件與依賴關係,以下將對各層進行詳細說明。

\begin{enumerate}

    \item Use Case 層:

    負責業務邏輯的處理,使用 Entity 作為資料容器。當 View Model 發起請求時,Use Case 會根據需求,調用 Repository 來獲得或操作數據。該層也定義了抽象的 Repository 介面。

    \item Repository 層:

    實作 Use Case 層定義的介面,負責從 Data Source 調用數據,取得 QueryItem 後轉換成 Entity 格式回傳給 Use Case。該層也定義了抽象的 Data Source 介面。

    \item Data Source 層:

    負責實際的數據存取。根據 Repository 的請求,Data Source 會去操作實際的數據來源,如資料庫、遠端服務或記憶體暫存等等,並以 QueryItem 作為回傳結果。

    \item View Model 層:

    該層為 MVVM 架構中的一部分,透過 Flutter 的 ChangeNotifier 實作。負責管理 UI 所需的狀態和邏輯。它會和 UI 進行雙向數據綁定,並透過 Use Case 處理業務邏輯和數據操作。

    \item UI 層:

    主要負責呈現使用者界面與互動。所有的 UI 元件都在這一層並組合成頁面,再透過 View Model 來獲得及顯示必要的數據。

\end{enumerate}

\subsection{資料定義}

我們定義了幾種核心資料:

\begin{enumerate}
    \item 旅程:代表一次旅遊的所有資料,包含多個軌跡段與旅遊資料
    \item 軌跡:使用者用GPS記錄的路徑
    \item 旅遊資料:照片、影片、錄音、文字等多媒體
    \item 座標點:用一組經緯度表示一個座標點
\end{enumerate}

\subsection{UI 介面設計}

為了使介面簡潔易用,我們設計了兩個主要頁面,及多個底部彈出面板(Bottom sheet),提供更高效率的操作體驗。圖~\ref{主要頁面與面板}是我們設計的主要頁面與面板。

\begin{itemize}
    \item 主要頁面:透過地圖(圖~\ref{地圖頁面})與圖庫(圖~\ref{圖庫頁面})頁面的切換,讓使用者能輕鬆查看目前的軌跡或旅遊資料,旅遊資料會根據其位置與時間順序被聚合成資料群集點,避免畫面過於雜亂。
    \item 同步資料顯示:透過主要頁面側邊的旅遊資料時間軸列表來同步兩個頁面所定位的資料群集。
    \item 快速紀錄:在地圖上方提供多功能輸入欄,能讓使用者在途中快速記錄文字筆記、拍攝照片或錄製語音。
    \item 底部彈出面板:讓使用者在不用切換頁面的情況下訪問各種功能,如獲取詳細資料(圖~\ref{旅程詳細資料面板})、與AI對話(圖~\ref{AI對話面板})、篩選資料(圖~\ref{旅遊資料篩選面板})與管理所有旅程(圖~\ref{旅程管理面板}),並能在主要頁面即時看到資料更新情況。
\end{itemize}

\newcommand{\customsubfig}[2]{
    \hspace{0.005\textwidth}
    \begin{subfigure}[t]{0.13\textwidth}
        \centering
        \includegraphics[width=\linewidth]{assets/#1.png}
        \caption{#2}
        \label{#2}
    \end{subfigure}
    \hspace{0.005\textwidth}
}

\begin{figure}[H]
    \centering
    \customsubfig{地圖頁面}{地圖頁面}
    \customsubfig{圖庫頁面}{圖庫頁面}
    \customsubfig{旅程詳細資料面板-統計資料}{旅程詳細資料面板}
    % \customsubfig{旅程詳細資料面板-軌跡列表}
    %
    % \vspace{0.5em}
    %
    % \customsubfig{AI對話面板}
    \customsubfig{AI對話面板-聊天}{AI對話面板}
    \customsubfig{旅遊資料篩選面板}{旅遊資料篩選面板}
    \customsubfig{旅程管理面板}{旅程管理面板}

    \caption{主要頁面與面板}
    \label{主要頁面與面板}
\end{figure}

\let\customsubfig\relax

% \begin{enumerate}

%     \item 地圖頁面:

%     如圖~\ref{地圖頁面},使用者可以在地圖上探索和回顧旅遊的軌跡與資料。所有的旅遊資料點將根據其地理位置與時間順序被聚合成資料群集點,以避免畫面過於複雜。側邊有一個與圖庫頁面共用的時間軸,可以同步兩個頁面要顯示的資料群集。上方的多功能輸入欄,能讓使用者在途中快速記錄文字筆記、拍攝照片或錄製語音。

%     \item 圖庫頁面:

%     如圖~\ref{圖庫頁面},以格狀排列展示所有的旅遊資料,包括圖片、筆記和語音記錄等。透過點擊側邊時間軸的資料群集,使用者能夠查看該點的所有旅遊資料。點擊旅遊資料會開啟詳細檢視視窗,並根據資料的不同類型呈現內容。

%     \item 旅程詳細資料面板:

%     如圖~\ref{旅程詳細資料面板},顯示與特定旅程相關的詳細統計資料。包括旅程的總時間、平均速度、總距離等。滑動至右側子頁面可以查看該旅程的所有軌跡資訊。

%     \item AI對話面板:

%     如圖~\ref{AI對話面板},使用者可以透過自然語言與AI助手進行對話,請求幫助或執行特定操作。例如,使用者可以請求AI助手根據地點或時間篩選旅遊資料、整理照片或提供旅程建議等。

%     \item 旅遊資料篩選面板:

%     如圖~\ref{旅遊資料篩選面板},提供一系列的篩選和排序選項,能根據標籤、地點、名稱、類別、時間等多個維度來找到和查看特定的旅遊資料。

%     \item 旅程管理面板:

%     如圖~\ref{旅程管理面板},提供一個集中的列表來管理記錄中和已完成的所有旅程。會標記目前正在記錄中的旅程,也可以選擇是否讓特定的旅程軌跡在地圖上顯示,以便於比較和回顧。

% \end{enumerate}


\section{結果}

我們的 APP 已經完成了初步的框架和核心功能,並具備了基本的用戶介面和旅程記錄、資料管理功能,並且成功入圍了「2023全國大專校院智慧創新暨跨域整合創作競賽」的決賽。這肯定了我們的技術實力,也鼓勵了我們繼續改進和完善產品,以期在決賽中取得好成績,並在未來能為更多的使用者提供便捷的旅遊記錄和管理解決方案。

% \section{參考文獻}

\end{document}