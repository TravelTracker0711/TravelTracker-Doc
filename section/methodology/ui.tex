\subsection{UI 介面設計}

為了使介面簡潔易用,我們設計了兩個主要頁面,及多個底部彈出面板(Bottom sheet),提供更高效率的操作體驗。圖~\ref{主要頁面與面板}是我們設計的主要頁面與面板。

\begin{itemize}
    \item 主要頁面:透過地圖(圖~\ref{地圖頁面})與圖庫(圖~\ref{圖庫頁面})頁面的切換,讓使用者能輕鬆查看目前的軌跡或旅遊資料,旅遊資料會根據其位置與時間順序被聚合成資料群集點,避免畫面過於雜亂。
    \item 同步資料顯示:透過主要頁面側邊的旅遊資料時間軸列表來同步兩個頁面所定位的資料群集。
    \item 快速紀錄:在地圖上方提供多功能輸入欄,能讓使用者在途中快速記錄文字筆記、拍攝照片或錄製語音。
    \item 底部彈出面板:讓使用者在不用切換頁面的情況下訪問各種功能,如獲取詳細資料(圖~\ref{旅程詳細資料面板})、與AI對話(圖~\ref{AI對話面板})、篩選資料(圖~\ref{旅遊資料篩選面板})與管理所有旅程(圖~\ref{旅程管理面板}),並能在主要頁面即時看到資料更新情況。
\end{itemize}

\newcommand{\customsubfig}[2]{
    \hspace{0.005\textwidth}
    \begin{subfigure}[t]{0.13\textwidth}
        \centering
        \includegraphics[width=\linewidth]{assets/#1.png}
        \caption{#2}
        \label{#2}
    \end{subfigure}
    \hspace{0.005\textwidth}
}

\begin{figure}[H]
    \centering
    \customsubfig{地圖頁面}{地圖頁面}
    \customsubfig{圖庫頁面}{圖庫頁面}
    \customsubfig{旅程詳細資料面板-統計資料}{旅程詳細資料面板}
    % \customsubfig{旅程詳細資料面板-軌跡列表}
    %
    % \vspace{0.5em}
    %
    % \customsubfig{AI對話面板}
    \customsubfig{AI對話面板-聊天}{AI對話面板}
    \customsubfig{旅遊資料篩選面板}{旅遊資料篩選面板}
    \customsubfig{旅程管理面板}{旅程管理面板}

    \caption{主要頁面與面板}
    \label{主要頁面與面板}
\end{figure}

\let\customsubfig\relax

% \begin{enumerate}

%     \item 地圖頁面:

%     如圖~\ref{地圖頁面},使用者可以在地圖上探索和回顧旅遊的軌跡與資料。所有的旅遊資料點將根據其地理位置與時間順序被聚合成資料群集點,以避免畫面過於複雜。側邊有一個與圖庫頁面共用的時間軸,可以同步兩個頁面要顯示的資料群集。上方的多功能輸入欄,能讓使用者在途中快速記錄文字筆記、拍攝照片或錄製語音。

%     \item 圖庫頁面:

%     如圖~\ref{圖庫頁面},以格狀排列展示所有的旅遊資料,包括圖片、筆記和語音記錄等。透過點擊側邊時間軸的資料群集,使用者能夠查看該點的所有旅遊資料。點擊旅遊資料會開啟詳細檢視視窗,並根據資料的不同類型呈現內容。

%     \item 旅程詳細資料面板:

%     如圖~\ref{旅程詳細資料面板},顯示與特定旅程相關的詳細統計資料。包括旅程的總時間、平均速度、總距離等。滑動至右側子頁面可以查看該旅程的所有軌跡資訊。

%     \item AI對話面板:

%     如圖~\ref{AI對話面板},使用者可以透過自然語言與AI助手進行對話,請求幫助或執行特定操作。例如,使用者可以請求AI助手根據地點或時間篩選旅遊資料、整理照片或提供旅程建議等。

%     \item 旅遊資料篩選面板:

%     如圖~\ref{旅遊資料篩選面板},提供一系列的篩選和排序選項,能根據標籤、地點、名稱、類別、時間等多個維度來找到和查看特定的旅遊資料。

%     \item 旅程管理面板:

%     如圖~\ref{旅程管理面板},提供一個集中的列表來管理記錄中和已完成的所有旅程。會標記目前正在記錄中的旅程,也可以選擇是否讓特定的旅程軌跡在地圖上顯示,以便於比較和回顧。

% \end{enumerate}