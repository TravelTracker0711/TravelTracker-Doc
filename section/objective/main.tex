\section{研究動機與目的}

在旅程途中我們常會使用照片、影片或是文字等載體保留旅程途中的美好回憶。但在旅遊中不會想花太多時間在記錄上,而且旅遊結束的資料量都很可觀,人工整理會耗費相當多的時間。

因此,我們想開發一款針對整理旅遊記錄的不便,提供使用者便捷的解決方案的APP。使用者可以快速記錄軌跡、影像與文字等旅遊資料,也能自動從手機載入影像,並根據影像的拍攝時間推算拍攝地點。讓使用者能更快速地找到旅遊途中的美好記錄。

% 此外,Travel Tracker也提供AI辨識的方式標記圖片類別,讓使用者能快速找到相對應的圖片。最後,Travel Tracker更結合AI對話技術,讓使用者可以輕鬆透過口語與AI對話,快速地整理旅遊中的資料。

% 我們推出的Travel Tracker主要希望解決旅遊記錄和整理的不便,提供快速的軌跡、影像與文字記錄功能。更重要的,我們結合AI辨識和對話技術,讓使用者能更輕鬆地整理和管理旅行中的美好時刻。

% 我們經過問卷調查與自身經驗,得出了旅遊中時常有以下的問題:

% 1. 旅途中花太多時間整理資料,或是照片太雜亂: 出去玩應該要把最多的時間放在享受過程,最少的時間在記錄與整理上,如果出去玩都在看著手機翻找照片,反而會本末倒置,失去去旅遊、體驗生活的意義。

% 2. 使用旅遊APP常常沒辦法自動匯入資料,需要一個一個加入: 旅遊過程會常常拍照,產生很多資料,如果要手動加入APP的話會花太多時間做重複性的事情,失去寶貴的時間,又沒辦法確認是否已經加入所有相關的照片。

% 3. 沒辦法支援所有想記錄的資料格式,如照片、文字、錄音等等:單純的相簿沒辦法存放文字,大部分日記APP又得自己慢慢插入圖片,沒辦法一次性將所有的資料整合起來,常常要在很多不同的位置找到自己的記錄。

% 4. 複雜的系統操作方式時常不夠直覺:旅遊過程如果需要花很多時間開啟APP,或是找到合適的位置記錄心中的想法,很可能會忘記本來旅途獲得的感受,也會用太多時間在解決系統問題上。
% 5. 簡單的系統提供的功能太陽春:與上一點相對,系統簡單雖然操作不複雜,但也不能支援所有整理的需求,反而會讓整理的過程變慢