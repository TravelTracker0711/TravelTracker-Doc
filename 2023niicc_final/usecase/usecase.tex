\newcounter{UCcounter}
\setcounter{UCcounter}{0}

\makeatletter
\newcommand{\autolabelUC}[1]{
  \stepcounter{UCcounter}
  \ifnum\value{UCcounter}<10
    \protected@edef\@currentlabel{TT-UC-0\arabic{UCcounter}}
  \else
    \protected@edef\@currentlabel{TT-UC-\arabic{UCcounter}}
  \fi
  \hspace*{-0.7em}
  \textbf{\@currentlabel}
  \label{#1}
}
\makeatother

\section{一般性的系統功能操作使用案例之劇本(Scenario)}

\begin{longtable}{|l|p{13.25cm}|}
  \hline
  \textbf{使用者案例} & \autolabelUC{軌跡和資料記錄} \\
  \hline
  \textbf{案例名稱} & 軌跡和資料記錄 \\
  \hline
  \textbf{相關功能性需求} & \ref{記錄軌跡}、\ref{暫停記錄軌跡}、\ref{從APP開啟相機}、\ref{從APP錄音}、\ref{記錄文字}、\newline \ref{快速記錄} \\
  \hline
  \textbf{使用者} & 一般使用者 \\
  \hline
  \textbf{前置條件} & 使用者已安裝相應的應用程式並準備記錄旅遊資料 \\
  \hline
  \textbf{說明} & 使用者可以在旅途中使用手機上應用程式記錄軌跡、拍照、錄音或文字,並可以使用快捷功能更快速的記錄資料,也可以隨時暫停記錄。 \\
  \hline
  \textbf{使用者操作} & 
  1. 使用者開啟記錄功能開始記錄軌跡,並有暫停鍵或結束鍵以提供暫停或結束記錄。\newline
  2. 使用者可以從APP中開啟手機上相機功能進行拍照或錄影。\newline
  3. 使用者可以從APP中開啟錄音功能開始錄音。\newline
  4. 使用者可以在APP中開啟文字記錄功能開始記錄文字。\newline
  5. 使用者可透過APP提供的桌面小工具或通知列選項,讓使用者可以不用打開APP就可以直接快速地記錄旅遊資料。 \\
  \hline
  \textbf{例外處理} & 若使用者的裝置沒有授予相關權限(例如鏡頭、錄音),則應提供適當的提示訊息。 \\
  \hline
  \textbf{測試方案} & 測試在沒有權限的情況下開啟相關功能,確認有提供提示訊息。 \\
  \hline
\end{longtable}

\begin{longtable}{|l|p{13.25cm}|}
  \hline
  \textbf{使用者案例} & \autolabelUC{檢視旅遊資料} \\
  \hline
  \textbf{案例名稱} & 檢視旅遊資料 \\
  \hline
  \textbf{相關功能性需求} & \ref{軌跡上顯示旅遊資料}、\ref{地圖旁以時間軸方式顯示旅遊資料}、\ref{切換圖庫顯示旅遊資料頁面}、\ref{檢視個別旅遊資料} \\
  \hline
  \textbf{使用者} & 一般使用者 \\
  \hline
  \textbf{前置條件} & 需將旅遊資料記錄好,包含行進軌跡、影像、錄音或記錄之文字。 \\
  \hline
  \textbf{說明} & 使用者可以依自己想瀏覽旅遊資料的形式選擇合適的旅遊資料瀏覽方式,瀏覽方式有:在地圖軌跡上顯示旅遊資料、在地圖旁以時間軸方式顯示對應軌跡的旅遊資料、格狀式顯示旅遊資料或條列式顯示旅遊資料,此外,也可以檢視個別或單一影像、錄音或文字。 \\
  \hline
  \textbf{描述(操作流程)} & 
  1. 使用者可直接在地圖頁籤縮放地圖來查看軌跡上的影像。\newline
  2. 使用者在地圖頁籤,可在地圖旁看到時間軸,以上下滑動的方式瀏覽旅遊資料,並可對應到地圖上的軌跡點。\newline
  3. 使用者可以向左滑,或是點擊資料頁籤,將時間軸上的資料展開檢視。\newline
  4. 使用者在資料頁籤時,從檢視模式選項選擇以格狀方式顯示旅遊資料。\newline
  5. 使用者在資料頁籤時,從檢視模式選項選擇以條列方式顯示旅遊資料。\newline
  6. 使用者可在任何模式點選個別旅遊資料並瀏覽。 \\
  \hline
  \textbf{例外處理} & 若使用者未記錄行進軌跡、影像、錄音或文字,則系統會顯示「無資料顯示」。 \\
  \hline
  \textbf{測試方案} & 使用者在未新增旅遊資料並點選瀏覽紀錄之功能時,確認系統是否顯示「無資料顯示」。 \\
  \hline
\end{longtable}

% \begin{longtable}{|l|p{13.25cm}|}
%   \hline
%   \textbf{使用者案例} & \textbf{TT-UC-03} \\
%   \hline
%   \textbf{案例名稱} & 軌跡管理 \\
%   \hline
%   \textbf{相關功能性需求} & 軌跡平滑化、拉伸軌跡、偵測軌跡停留位置並簡化 \\
%   \hline
%   \textbf{使用者} & 一般使用者 \\
%   \hline
%   \textbf{前置條件} & 使用者已記錄完軌跡。 \\
%   \hline
%   \textbf{說明} & 使用者可以對已經記錄的軌跡進行平滑化、拉伸或是偵測軌跡停留位置並簡化。 \\
%   \hline
%   \textbf{描述(操作流程)} & 
%   1. 使用者選擇已記錄的軌跡。\newline
%   2. 使用者選擇平滑化軌跡。\newline
%   3. 使用者選擇拉伸軌跡。\newline
%   4. 若平滑化軌跡後軌跡有雜亂的點,使用者可以選擇偵測軌跡停留位置並將雜亂的軌跡路線簡化成一個點。 \\
%   \hline
%   \textbf{例外處理} & 使用者沒有記錄軌跡,則系統應提示「無軌跡」。 \\
%   \hline
%   \textbf{測試方案} & 測試不輸入軌跡,確保所有修改功能對使用者顯示對應之提示。 \\
%   \hline
% \end{longtable}

\begin{longtable}{|l|p{13.25cm}|}
  \hline
  \textbf{使用者案例} & \autolabelUC{旅遊資料管理} \\
  \hline
  \textbf{案例名稱} & 旅遊資料管理 \\
  \hline
  \textbf{相關功能性需求} & \ref{編輯旅遊資料的標籤}、\ref{編輯旅遊資料的日期}、\ref{刪除旅遊資料}、\ref{與AI對話來選取並編輯與刪除旅遊資料}、\ref{AI自動辨識並標記照片} \\
  \hline
  \textbf{使用者} & 一般使用者 \\
  \hline
  \textbf{前置條件} & 使用者已記錄旅遊資料(影像、錄音或文字) \\
  \hline
  \textbf{說明} & 使用者可以編輯標籤、日期,刪除資料,或批量處理。此外,還可以透過AI的協助進行更智能的資料管理,例如自動辨識標記照片並在搜尋欄中搜尋已標記的照片。 \\
  \hline
  \textbf{使用者操作} & 
  1. 使用者選擇旅遊資料。\newline
  2. 使用者編輯標籤和日期。\newline
  3. 使用者刪除旅遊資料。\newline
  4. 使用者與AI對話來選取、編輯或刪除資料。\newline
  5. AI自動辨識並標記照片。\newline
  6. 使用者在搜尋欄中搜尋已標記的照片。 \\
  \hline
  \textbf{例外處理} & 若資料缺失、系統無法正確解析或AI無法辨別標記照片,則應提供適當的錯誤訊息。 \\
  \hline
  \textbf{測試方案} & 
  1. 使用者在未新增旅遊資料時,確認系統是否顯示「無資料」。\newline
  2. 測試將照片放置系統中,AI是否可以自動辨認並標記照片。\newline
  3. 測試在搜尋欄中搜尋已標記的內容,確認系統是否可顯示相對應的內容。 \\
  \hline
\end{longtable}