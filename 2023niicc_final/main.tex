\documentclass[12pt]{article}

% 頁面設置
\usepackage[right=20mm, bottom=30mm, left=20mm]{geometry} % 頁面邊界
% \usepackage{multicol} % 頁面分欄
% \setlength{\columnsep}{1pt} % 欄間距
\usepackage{float} % 浮動物件 (圖表 H)

% 字體設定
\usepackage{type1cm} % 設定字體大小
\usepackage{xeCJK} % 中文字體
% \setCJKmainfont{Noto Sans TC}
\ifdefined\nosystemfont
    \setCJKmainfont[
        Path=../font/,
        UprightFont={kaiu.ttf}
    ]{標楷體}
\else
    \setCJKmainfont{kaiu.ttf} % 標楷體
\fi

% 設定首行縮排
\newlength{\fullwidthspace}
\setlength{\fullwidthspace}{2em}  % 定義兩個全形空白的長度為2em
\setlength{\parindent}{\fullwidthspace}  % 設定段落首行縮排為兩個全形空白

% 套件區
\usepackage{enumitem} % 列表 (enumerate, itemize)
\usepackage{array} % 表格
\usepackage{makecell} % 表格換行

\usepackage{longtable} % 長表格
\usepackage{indentfirst} % 自動首行縮排
\usepackage{fancyhdr} % 頁首頁尾
\usepackage{lastpage} % 最後一頁的頁數
\renewcommand{\arraystretch}{1.45} % 表格行高
\usepackage{collcell}
\usepackage{etoolbox} % 用 gappto 累積一個指令
\usepackage{keyval} % 參數的 key-value 對

\usepackage{graphicx} % 圖片
\usepackage{caption} % 圖片標題
\usepackage{subcaption} % 圖片子標題
\captionsetup[figure]{name={圖},labelsep=period}

% 使用 hyperref 並設定
\usepackage[unicode=true,pdfusetitle,
 bookmarks=true,bookmarksnumbered=false,bookmarksopen=false,
 breaklinks=false,pdfborder={0 0 0},backref=true,colorlinks=false]
 {hyperref}

% \usepackage{amssymb} % 數學符號
% \usepackage[fleqn]{amsmath} % 數學排版, fleqn 選項讓數學式靠左對齊
% \usepackage{tikz} % 繪圖
% \usepackage{pgfplots} % 圖表
% \usepackage{caption} % 圖表標題
% \usepackage{subcaption} % 圖表子標題
% \usepackage{subfig} % 圖表子圖

% tikz 設定
% \tikzset{every state, accepting/.style={double distance=2pt}}
% \usetikzlibrary{automata, positioning, arrows}

% fancyhdr 設定
\pagestyle{fancy}
\renewcommand{\footnotesize}{\normalsize} % 設定腳註字型大小
\renewcommand{\headrulewidth}{0pt} % 不要有頁首橫線
\renewcommand{\footrulewidth}{0pt} % 不要有頁尾橫線
\lhead{}
\chead{2023年全國大專校院智慧創新暨跨域整合創作競賽 - 作品測試文件}
\rhead{}
\lfoot{}
\cfoot{}
\rfoot{ 共 \pageref{LastPage} 頁 第 \thepage 頁} 

\begin{document}

\tableofcontents % 目錄
\newpage

% 標題
% \title{}
% \author{}
% \date{}
% \maketitle

% 系統目的與範圍、提供服務與技術
\noindent
\textbf{參賽隊名:} TraTracker \\
\textbf{作品名稱:} 旅遊紀錄整理工具 \\
\textbf{系統名稱:} Travel Tracker

\section{系統目的與範圍}

Travel Tracker 的主要目的是解決旅遊記錄、整理的不便,創造一個便捷的方式在旅程中輕鬆地記錄和整理旅遊資料。通過使用地圖、AI對話和自動標籤等技術,讓使用者能快速找到和整理旅遊途中的美好記錄。

\section{Travel Tracker 提供的服務}
\begin{enumerate}
  \item \textbf{快速記錄軌跡與多媒體資料}:不用花時間開啟各種程式,能以最快的速度記錄回憶。
  \item \textbf{自動影像載入與地點推算}:自動從手機載入影像,並根據拍攝時間與軌跡資訊推算拍攝地點。
  \item \textbf{以拍攝地點分組旅遊資料}:利用地理資訊,讓使用者依照地點快速找到旅遊回憶。
  \item \textbf{AI辨識標記與對話整理}:使用AI技術自動標記圖像類別,並透過AI對話技術輕鬆整理旅遊資料。
  \item \textbf{全方位多媒體支援}:同時支援照片、影片、錄音、文字等資料形式,提供多元化的記錄方式。
\end{enumerate}

\section{Travel Tracker 涵蓋的技術}
\begin{enumerate}
  \item \textbf{地圖軌跡整合技術}:使用地圖作為主要界面,結合GPS訊號和拍攝地點的推算,使旅遊資料更具關聯性和直觀性。
  \item \textbf{人工智慧辨識與對話技術}:利用AI分析照片內容進行自動標記,並結合自然語言處理技術,讓使用者能夠與AI進行對話,快速整理旅遊資料。
  \item \textbf{跨平台兼容性}:支援不同的移動設備,透過適應性設計讓使用者在不同設備上都能夠方便地使用。
\end{enumerate}


% 非功能性需求
\newcounter{nfcounter}
\setcounter{nfcounter}{0}

\makeatletter
\newcommand{\autolabelNF}[1]{
  \stepcounter{nfcounter}
  \ifnum\value{nfcounter}<10
    \protected@edef\@currentlabel{TT-NF-0\arabic{nfcounter}}
  \else
    \protected@edef\@currentlabel{TT-NF-\arabic{nfcounter}}
  \fi
  \hspace*{-0.7em}
  \@currentlabel
  \label{#1}
}
\makeatother

\section{系統非功能需求}
\begin{longtable}{|c|p{5.2cm}|p{7.5cm}|}
  \hline
  \textbf{非功能需求編號} & \textbf{功能名稱} & \textbf{功能需求描述} \\
  \hline
  \endfirsthead
  \hline
  \textbf{非功能需求編號} & \textbf{功能名稱} & \textbf{功能需求描述} \\
  \hline
  \endhead
  \autolabelNF{快速記錄} & 快速記錄 & 桌面小工具要取得快速、操作簡單明瞭方便 \\
  \hline
  \autolabelNF{大型資料處理效率} & 大型資料處理效率 & 系統要能處理大型媒體檔案不會當機的效能 \\
  \hline
  \autolabelNF{使用者界面} & 使用者界面 & 系統界面必須直覺易用,方便使用者操作和理解。視覺設計需統一並符合現代化的美學標準。 \\
  \hline
  \autolabelNF{可用性} & 可用性 & 系統應具備高可用性,保證穩定的運行。所有功能都應在普通網絡條件下迅速響應。 \\
  \hline
  \autolabelNF{安全性} & 安全性 & 所有個人和旅遊相關資料必須安全存儲,並符合相關隱私法規。應用程式應具有適當的權限管理,防止未經授權的訪問。 \\
  \hline
  \autolabelNF{兼容性} & 兼容性 & 應用程式應支援主流的移動操作系統和版本。地圖和多媒體資料應在不同的裝置和解析度上正常顯示。 \\
  \hline
  \autolabelNF{地圖視圖控制} & 地圖視圖控制 & 系統應支援不同地圖主題和樣式的切換。定位功能必須精確且迅速反應。 \\
  \hline
  \autolabelNF{資料匯入、匯出} & 資料匯入、匯出 & 資料匯入和匯出應支援通用格式,如CSV、GPX等。匯入功能應能自動識別和配對相關媒體和軌跡。 \\
  \hline
  \autolabelNF{旅程管理} & 旅程管理 & 旅程切換和管理應流暢且直覺,支援多旅程同時記錄和顯示。提供完整的旅程統計資訊,如總距離、時間等。 \\
  \hline
  \autolabelNF{桌面小工具} & 桌面小工具 & 桌面小工具需提供快速記錄和查看功能,並且操作簡單明瞭方便。 \\
  \hline
  \autolabelNF{AI 功能} & AI 功能 & AI對話和自動辨識功能應準確並能有效協助使用者進行篩選、編輯等操作。 \\
  \hline
\end{longtable}

% 功能性需求
\section{系統功能需求}

\begin{itemize}
  \item 旅程:包含一次旅遊的所有軌跡、旅遊資料
  \item 軌跡:使用者用GPS記錄的路徑
  \item 旅遊資料:照片、影片、錄音、文字等多媒體
\end{itemize}

%記錄功能 RC
\newcounter{RCcounter}
\setcounter{RCcounter}{0}

\makeatletter
\newcommand{\autolabelRC}[1]{
  \stepcounter{RCcounter}
  \ifnum\value{RCcounter}<10
    \protected@edef\@currentlabel{TT-RC-0\arabic{RCcounter}}
  \else
    \protected@edef\@currentlabel{TT-RC-\arabic{RCcounter}}
  \fi
  \hspace*{-0.7em}
  \@currentlabel
  \label{#1}
}
\makeatother

\subsection{記錄功能 RC}

\begin{longtable}{|c|p{4.3cm}|p{8.9cm}|}
  \hline
  \textbf{功能需求編號} & \textbf{功能名稱} & \textbf{功能需求描述} \\
  \hline
  \endfirsthead
  \hline
  \textbf{功能需求編號} & \textbf{功能名稱} & \textbf{功能需求描述} \\
  \hline
  \endhead
  \autolabelRC{記錄軌跡} & 記錄軌跡 & 使用者可以通過GPS記錄旅程軌跡 \\
  \hline
  \autolabelRC{暫停記錄軌跡} & 暫停記錄軌跡 & 允許使用者暫停記錄旅程軌跡 \\
  \hline
  \autolabelRC{從APP開啟相機} & 從APP開啟相機 & 使用者可以直接從應用程式開啟相機進行拍照 \\
  \hline
  \autolabelRC{從APP錄音} & 從APP錄音 & 使用者可以直接從應用程式開啟麥克風進行錄音 \\
  \hline
  \autolabelRC{記錄文字} & 記錄文字 & 允許使用者記錄文字內容,如日記或心得 \\
  \hline
  \autolabelRC{快速記錄} & 快速記錄 & 使用者可以通過桌面小工具或通知列快速記錄文字、錄音等等 \\
  \hline
  % TODO
  \autolabelRC{自動抓取媒體資料} & 自動抓取媒體資料 & 自動從手機的相簿或其他媒體資料夾中抓取相關旅遊媒體資料 \\
  \hline
\end{longtable}

%檢視軌跡與旅遊資料 RV
\newAutoLabel{TT-RV}

\subsection{檢視軌跡與旅遊資料 RV}

\begin{longtable}{|c|p{4.3cm}|p{8.9cm}|}
  \hline
  \textbf{功能需求編號} & \textbf{功能名稱} & \textbf{功能需求描述} \\
  \hline
  \endfirsthead
  \hline
  \textbf{功能需求編號} & \textbf{功能名稱} & \textbf{功能需求描述} \\
  \hline
  \endhead
  \autoLabel{地圖上顯示軌跡} & 地圖上顯示軌跡 & 在地圖上展示使用者的旅程軌跡 \\
  \hline
  \autoLabel{檢視旅程、軌跡的統計資料} & 檢視旅程、軌跡的統計資料 & 檢視總距離、時間、速度等統計資料 \\
  \hline
  \autoLabel{軌跡上顯示旅遊資料} & 軌跡上顯示旅遊資料 & 軌跡上顯示旅遊相關資料,如照片、影片、文字等 \\
  \hline
  \autoLabel{軌跡上旅遊資料會以群集顯示} & 軌跡上旅遊資料會以群集顯示 & 將地圖上距離近的旅遊資料集中成一個群集 \\
  \hline
  \autoLabel{地圖旁以時間軸方式顯示旅遊資料} & 地圖旁以時間軸方式顯示旅遊資料 & 旅遊資料以時間軸形式展示,方便查看 \\
  \hline
  \autoLabel{切換圖庫顯示旅遊資料頁面} & 切換圖庫顯示旅遊資料頁面 & 使用者能切換至圖庫頁面檢視旅遊資料 \\
  \hline
  \autoLabel{檢視個別旅遊資料} & 檢視個別旅遊資料 & 查看單一旅遊資料的詳細資訊 \\
  \hline
  \autoLabel{利用標籤、日期等項目篩選旅遊資料} & 利用標籤、日期等項目篩選旅遊資料 & 根據日期、標籤等條件篩選和搜尋旅遊資料 \\
  \hline
  \autoLabel{用地理位置篩選旅遊資料} & 用地理位置篩選旅遊資料 & 一次選取某個位置周圍的旅遊資料 \\
  \hline
  \autoLabel{與AI對話來篩選旅遊資料} & 與AI對話來篩選旅遊資料 & 通過AI對話界面協助篩選旅遊資料 \\
  \hline
\end{longtable}

%編輯軌跡與旅遊資料 WR
\newcounter{WRcounter}
\setcounter{WRcounter}{0}

\makeatletter
\newcommand{\autolabelWR}[1]{
  \stepcounter{WRcounter}
  \ifnum\value{WRcounter}<10
    \protected@edef\@currentlabel{TT-WR-0\arabic{WRcounter}}
  \else
    \protected@edef\@currentlabel{TT-WR-\arabic{WRcounter}}
  \fi
  \hspace*{-0.7em}
  \@currentlabel
  \label{#1}
}
\makeatother

\subsection{編輯軌跡與旅遊資料 WR}

\begin{longtable}{|c|p{4.3cm}|p{8.9cm}|}
  \hline
  \textbf{功能需求編號} & \textbf{功能名稱} & \textbf{功能需求描述} \\
  \hline
  \endfirsthead
  \hline
  \textbf{功能需求編號} & \textbf{功能名稱} & \textbf{功能需求描述} \\
  \hline
  \endhead
  \autolabelWR{軌跡平滑化} & 軌跡平滑化 & 將記錄的軌跡進行平滑處理,使之更符合實際路線 \\
  \hline
  \autolabelWR{拉伸軌跡} & 拉伸軌跡 & 允許使用者手動調整軌跡的形狀 \\
  \hline
  \autolabelWR{偵測軌跡停留位置並簡化} & 偵測軌跡停留位置並簡化 & 能自動識別軌跡中的停留位置,並將其 GPS 誤差造成的一團路徑簡化為一個點 \\
  \hline
  \autolabelWR{編輯旅遊資料的標籤} & 編輯旅遊資料的標籤 & 允許使用者為旅遊資料添加或修改標籤 \\
  \hline
  \autolabelWR{編輯旅遊資料的日期} & 編輯旅遊資料的日期 & 允許使用者為旅遊資料添加或修改日期 \\
  \hline
  \autolabelWR{刪除旅遊資料} & 刪除旅遊資料 & 允許使用者刪除指定的旅遊資料 \\
  \hline
  \autolabelWR{批量處理旅遊資料} & 批量處理旅遊資料 & 選取多個旅遊資料並一起編輯或刪除 \\
  \hline
  \autolabelWR{與AI對話來選取並編輯與刪除旅遊資料} & 與AI對話來選取並編輯與刪除旅遊資料 & 通過AI對話協助選取、編輯或刪除旅遊資料 \\
  \hline
  \autolabelWR{AI自動辨識並標記照片} & AI自動辨識並標記照片 & AI會自動分析並標記照片中的物體或景物 \\
  \hline
\end{longtable}

% 地圖視圖控制 MC
\newcounter{MCcounter}
\setcounter{MCcounter}{0}

\makeatletter
\newcommand{\autolabelMC}[1]{
  \stepcounter{MCcounter}
  \ifnum\value{MCcounter}<10
    \protected@edef\@currentlabel{TT-F-MC-0\arabic{MCcounter}}
  \else
    \protected@edef\@currentlabel{TT-F-MC-\arabic{MCcounter}}
  \fi
  \hspace*{-0.7em}
  \@currentlabel
  \label{#1}
}
\makeatother

\subsection{地圖視圖控制 MC}

\begin{longtable}{|c|p{4.3cm}|p{8.9cm}|}
  \hline
  \textbf{功能需求編號} & \textbf{功能名稱} & \textbf{功能需求描述} \\
  \hline
  \endfirsthead
  \hline
  \textbf{功能需求編號} & \textbf{功能名稱} & \textbf{功能需求描述} \\
  \hline
  \endhead
  \autolabelMC{基礎控制動作} & 基礎控制動作 & 使用者可以縮放、移動、旋轉地圖 \\
  \hline
  \autolabelMC{定位至目前位置} & 定位至目前位置 & 讓使用者能快速定位到當前位置 \\
  \hline
  \autolabelMC{定位至指定旅遊資料} & 定位至指定旅遊資料 & 允許使用者快速定位至特定旅遊資料 \\
  \hline
  \autolabelMC{定位至指定軌跡} & 定位至指定軌跡 & 允許使用者快速定位至特定軌跡 \\
  \hline
  \autolabelMC{地圖樣式選擇} & 地圖樣式選擇 & 使用者可以選擇不同的地圖主題和樣式,如魯地圖、OpenStreetMap、夜間模式、衛星圖像等 \\
  \hline
  \autolabelMC{離線地圖} & 離線地圖 & 使用者可以下載地圖,以便在沒有網路的情況下使用 \\
  \hline
\end{longtable}

% 資料匯入、匯出 EP
\newcounter{EPcounter}
\setcounter{EPcounter}{0}

\makeatletter
\newcommand{\autolabelEP}[1]{
  \stepcounter{EPcounter}
  \ifnum\value{EPcounter}<10
    \protected@edef\@currentlabel{TT-F-EP-0\arabic{EPcounter}}
  \else
    \protected@edef\@currentlabel{TT-F-EP-\arabic{EPcounter}}
  \fi
  \hspace*{-0.7em}
  \@currentlabel
  \label{#1}
}
\makeatother

\subsection{資料匯入、匯出 EP}

\begin{longtable}{|c|p{4.3cm}|p{8.9cm}|}
  \hline
  \textbf{功能需求編號} & \textbf{功能名稱} & \textbf{功能需求描述} \\
  \hline
  \endfirsthead
  \hline
  \textbf{功能需求編號} & \textbf{功能名稱} & \textbf{功能需求描述} \\
  \hline
  \endhead
  \autolabelEP{匯出旅程、軌跡、旅遊資料} & 匯出旅程、軌跡、旅遊資料 & 允許使用者將各式資料匯出為特定格式的檔案 \\
  \hline
  \autolabelEP{匯入旅程、軌跡、旅遊資料} & 匯入旅程、軌跡、旅遊資料 & 允許使用者將各式資料匯入,並且能根據軌跡的時間戳記從手機內自動讀取相應的媒體 \\
  \hline
  \autolabelEP{匯出軌跡圖} & 匯出軌跡圖 & 允許使用者產生整個軌跡在地圖上的截圖 \\
  \hline
  \autolabelEP{分享到社群軟體} & 分享到社群軟體 & 允許使用者選擇照片,並將照片分享到社群軟體 \\
  \hline
\end{longtable}

% 旅程管理 TM
\newAutoLabel{TT-TM}

\subsection{旅程管理 TM}
\begin{longtable}{|c|p{4.3cm}|p{8.9cm}|}
  \hline
  \textbf{功能需求編號} & \textbf{功能名稱} & \textbf{功能需求描述} \\
  \hline
  \endfirsthead
  \hline
  \textbf{功能需求編號} & \textbf{功能名稱} & \textbf{功能需求描述} \\
  \hline
  \endhead
  \autoLabel{創建新旅程} & 創建新旅程 & 允許使用者創建新的旅程 \\
  \hline
  \autoLabel{切換目前記錄旅程} & 切換目前記錄旅程\footnote[1] & 允許使用者切換正在記錄的旅程 \\
  \hline
  \autoLabel{切換各個旅程的可見性} & 切換各個旅程的可見性 & 允許使用者切換各個旅程的可見性以方便查看 \\
  \hline
\end{longtable}

\footnotetext[1]{顯示中與記錄中的旅程是獨立的,為了讓使用者可以在記錄B旅遊時回去看A旅遊的資料}

  

% 使用案例
\section{一般性的系統功能操作使用案例之劇本(Scenario)}

\begin{longtable}{|l|p{13.25cm}|}
  \hline
  \textbf{使用者案例} & \textbf{TT-UC-01} \\
  \hline
  \textbf{案例名稱} & 軌跡和資料記錄 \\
  \hline
  \textbf{相關功能性需求} & \ref{記錄軌跡}、\ref{暫停記錄軌跡}、\ref{從APP開啟相機}、\ref{從APP錄音}、\ref{記錄文字}、\newline \ref{快速記錄} \\
  \hline
  \textbf{使用者} & 一般使用者 \\
  \hline
  \textbf{前置條件} & 使用者已安裝相應的應用程式並準備記錄旅遊資料 \\
  \hline
  \textbf{說明} & 使用者可以在旅途中使用手機上應用程式記錄軌跡、拍照、錄音或文字,並可以使用快捷功能更快速的記錄資料,也可以隨時暫停記錄。 \\
  \hline
  \textbf{使用者操作} & 
  1. 使用者開啟記錄功能開始記錄軌跡,並有暫停鍵或結束鍵以提供暫停或結束記錄。\newline
  2. 使用者可以從APP中開啟手機上相機功能進行拍照或錄影。\newline
  3. 使用者可以從APP中開啟錄音功能開始錄音。\newline
  4. 使用者可以在APP中開啟文字記錄功能開始記錄文字。\newline
  5. 使用者可透過APP提供的桌面小工具或通知列選項,讓使用者可以不用打開APP就可以直接快速地記錄旅遊資料。 \\
  \hline
  \textbf{例外處理} & 若使用者的裝置沒有授予相關權限(例如鏡頭、錄音),則應提供適當的提示訊息。 \\
  \hline
  \textbf{測試方案} & 測試在沒有權限的情況下開啟相關功能,確認有提供提示訊息。 \\
  \hline
\end{longtable}

\begin{longtable}{|l|p{13.25cm}|}
  \hline
  \textbf{使用者案例} & \textbf{TT-UC-02} \\
  \hline
  \textbf{案例名稱} & 檢視旅遊資料 \\
  \hline
  \textbf{相關功能性需求} & 軌跡上顯示旅遊資料、地圖旁以時間軸方式顯示旅遊資料、格狀顯示旅遊資料、條列式詳細顯示旅遊資料、檢視個別旅遊資料 \\
  \hline
  \textbf{使用者} & 一般使用者 \\
  \hline
  \textbf{前置條件} & 需將旅遊資料記錄好,包含行進軌跡、影像、錄音或記錄之文字。 \\
  \hline
  \textbf{說明} & 使用者可以依自己想瀏覽旅遊資料的形式選擇合適的旅遊資料瀏覽方式,瀏覽方式有:在地圖軌跡上顯示旅遊資料、在地圖旁以時間軸方式顯示對應軌跡的旅遊資料、格狀式顯示旅遊資料或條列式顯示旅遊資料,此外,也可以檢視個別或單一影像、錄音或文字。 \\
  \hline
  \textbf{描述(操作流程)} & 
  1. 使用者可直接在地圖頁籤縮放地圖來查看軌跡上的影像。\newline
  2. 使用者在地圖頁籤,可在地圖旁看到時間軸,以上下滑動的方式瀏覽旅遊資料,並可對應到地圖上的軌跡點。\newline
  3. 使用者可以向左滑,或是點擊資料頁籤,將時間軸上的資料展開檢視。\newline
  4. 使用者在資料頁籤時,從檢視模式選項選擇以格狀方式顯示旅遊資料。\newline
  5. 使用者在資料頁籤時,從檢視模式選項選擇以條列方式顯示旅遊資料。\newline
  6. 使用者可在任何模式點選個別旅遊資料並瀏覽。 \\
  \hline
  \textbf{例外處理} & 若使用者未記錄行進軌跡、影像、錄音或文字,則系統會顯示「無資料顯示」。 \\
  \hline
  \textbf{測試方案} & 使用者在未新增旅遊資料並點選瀏覽紀錄之功能時,確認系統是否顯示「無資料顯示」。 \\
  \hline
\end{longtable}

\begin{longtable}{|l|p{13.25cm}|}
  \hline
  \textbf{使用者案例} & \textbf{TT-UC-03} \\
  \hline
  \textbf{案例名稱} & 軌跡管理 \\
  \hline
  \textbf{相關功能性需求} & 軌跡平滑化、拉伸軌跡、偵測軌跡停留位置並簡化 \\
  \hline
  \textbf{使用者} & 一般使用者 \\
  \hline
  \textbf{前置條件} & 使用者已記錄完軌跡。 \\
  \hline
  \textbf{說明} & 使用者可以對已經記錄的軌跡進行平滑化、拉伸或是偵測軌跡停留位置並簡化。 \\
  \hline
  \textbf{描述(操作流程)} & 
  1. 使用者選擇已記錄的軌跡。\newline
  2. 使用者選擇平滑化軌跡。\newline
  3. 使用者選擇拉伸軌跡。\newline
  4. 若平滑化軌跡後軌跡有雜亂的點,使用者可以選擇偵測軌跡停留位置並將雜亂的軌跡路線簡化成一個點。 \\
  \hline
  \textbf{例外處理} & 使用者沒有記錄軌跡,則系統應提示「無軌跡」。 \\
  \hline
  \textbf{測試方案} & 測試不輸入軌跡,確保所有修改功能對使用者顯示對應之提示。 \\
  \hline
\end{longtable}

\begin{longtable}{|l|p{13.25cm}|}
  \hline
  \textbf{使用者案例} & \textbf{TT-UC-04} \\
  \hline
  \textbf{案例名稱} & 旅遊資料管理 \\
  \hline
  \textbf{相關功能性需求} & 編輯旅遊資料的標籤、編輯旅遊資料的日期、刪除旅遊資料、批量處理旅遊資料、與AI對話來選取並編輯與刪除旅遊資料、AI自動辨識並標記照片 \\
  \hline
  \textbf{使用者} & 一般使用者 \\
  \hline
  \textbf{前置條件} & 使用者已記錄旅遊資料(影像、錄音或文字) \\
  \hline
  \textbf{說明} & 使用者可以編輯標籤、日期,刪除資料,或批量處理。此外,還可以透過AI的協助進行更智能的資料管理,例如自動辨識標記照片並在搜尋欄中搜尋已標記的照片。 \\
  \hline
  \textbf{使用者操作} & 
  1. 使用者選擇旅遊資料。\newline
  2. 使用者編輯標籤和日期。\newline
  3. 使用者刪除旅遊資料。\newline
  4. 使用者批量處理旅遊資料。\newline
  5. 使用者與AI對話來選取、編輯或刪除資料。\newline
  6. AI自動辨識並標記照片。\newline
  7. 使用者在搜尋欄中搜尋已標記的照片。 \\
  \hline
  \textbf{例外處理} & 若資料缺失、系統無法正確解析或AI無法辨別標記照片,則應提供適當的錯誤訊息。 \\
  \hline
  \textbf{測試方案} & 
  1. 使用者在未新增旅遊資料時,確認系統是否顯示「無資料」。\newline
  2. 測試將照片放置系統中,AI是否可以自動辨認並標記照片。\newline
  3. 測試在搜尋欄中搜尋已標記的內容,確認系統是否可顯示相對應的內容。 \\
  \hline
\end{longtable}

% 系統架構設計
\section{系統架構設計}

% 系統介面設計
\section{系統介面設計}

% 軟體或硬體架構設計
\section{軟體或硬體架構設計}

\subsection{資產 AssetItem}

\subsubsection{資產定義與說明}

\begin{longtable}{|c|c|c|c|}
  \hline
  \textbf{欄位名稱} & \textbf{欄位代號} & \textbf{定義} &\textbf{型態} \\
  \hline
  \endfirsthead
  \hline
  \textbf{欄位名稱} & \textbf{欄位代號} & \textbf{定義} &\textbf{型態} \\
  \hline
  \endhead
  識別碼 & id & 資料的唯一識別碼 & String \\
  \hline
  名稱 & name & 資產的名稱 & String \\
  \hline
  描述 & description & 資產的詳細說明 & String \\
  \hline
  標籤 & tags & 資產相關的標籤,用於分類和搜尋 & Set<String> \\
  \hline
  建立時間 & createdTime & 資產建立的時間 & DateTime \\
  \hline
  更新時間 & updatedTime & 資產最後一次被更新的時間 & DateTime \\
  \hline
  類型 & type & 資產的類型(如:影像、文件等) & String \\
  \hline
  管理識別碼 & relatedId & 使用或管理該資產的資源識別碼 & String \\
  \hline
\end{longtable}

\subsubsection{資產範例}

\begin{longtable}{|c|c|c|}
  \hline
  \textbf{欄位名稱} & \textbf{範例 1} & \textbf{範例 2} \\
  \hline
  \endfirsthead
  \hline
  \textbf{欄位名稱} & \textbf{範例 1} & \textbf{範例 2} \\
  \hline
  \endhead
  
  id & ASSET101 & ASSET102 \\
  \hline
  name & 西門町 & 士林夜市 \\
  \hline
  description & 台北市熱鬧的購物和娛樂區 & 台北著名的大型夜市 \\
  \hline
  tags & \{"人潮", "娛樂", "觀光"\} & \{"夜市", "美食"\} \\
  \hline
  createdTime & 2023-10-10 09:30:00 & 2023-10-11 10:45:00 \\
  \hline
  updatedTime & 2023-10-10 15:00:00 & 2023-10-11 18:00:00 \\
  \hline
  type & image & image \\
  \hline
  relatedId & TRACK002 & TRACK002 \\
  \hline
\end{longtable}


\subsection{旅程 TravelTrackItem}

\subsubsection{旅程定義與說明}

\begin{longtable}{|c|c|c|c|}
  \hline
  \textbf{欄位名稱} & \textbf{欄位代號} & \textbf{定義} & \textbf{型態} \\
  \hline
  \endfirsthead
  \hline
  \textbf{欄位名稱} & \textbf{欄位代號} & \textbf{定義} & \textbf{型態} \\
  \hline
  \endhead
  
  識別碼 & id & 用於唯一識別該旅行路線項目 & String \\
  \hline
  名稱 & name & 旅行路線的名稱或標題 & String \\
  \hline
  描述 & description & 旅行路線的詳細說明或簡介 & String \\
  \hline
  標籤 & tags & 與旅行路線相關的關鍵詞或分類 & Set<String> \\
  \hline
  建立時間 & createdTime & 該旅行路線項目的建立日期與時間 & DateTime \\
  \hline
  更新時間 & updatedTime & 該旅行路線項目最後一次更新的日期與時間 & DateTime \\
  \hline
\end{longtable}

\subsubsection{旅程範例}

\begin{longtable}{|c|c|c|}
  \hline
  \textbf{欄位名稱} & \textbf{範例 1} & \textbf{範例 2} \\
  \hline
  \endfirsthead
  \hline
  \textbf{欄位名稱} & \textbf{範例 1} & \textbf{範例 2} \\
  \hline
  \endhead
  
  id & TRACK001 & TRACK002 \\
  \hline
  name & 環島自由行 & 台北一日遊 \\
  \hline
  description & 台灣環島七日遊,體驗不一樣的風光 & 在台北市區走走逛逛,一天內感受都市魅力 \\
  \hline
  tags & \{"環島", "七日", "自由行"\} & \{"台北", "一日遊", "市區"\} \\
  \hline
  createdTime & 2023-10-01 09:00:00 & 2023-10-15 08:30:00 \\
  \hline
  updatedTime & 2023-10-07 21:00:00 & 2023-10-15 20:00:00 \\
  \hline
\end{longtable}


\subsection{軌跡 TrksegItem}

\subsubsection{軌跡定義與說明}

\begin{longtable}{|c|c|c|c|}
  \hline
  \textbf{欄位名稱} & \textbf{欄位代號} & \textbf{定義} & \textbf{型態} \\
  \hline
  \endfirsthead
  \hline
  \textbf{欄位名稱} & \textbf{欄位代號} & \textbf{定義} & \textbf{型態} \\
  \hline
  \endhead
  
  識別碼 & id & 用於唯一識別該路段項目 & String \\
  \hline
  名稱 & name & 路段的名稱或標題 & String \\
  \hline
  描述 & description & 路段的詳細說明或簡介 & String \\
  \hline
  標籤 & tags & 與該路段相關的關鍵詞或分類 & Set<String> \\
  \hline
  建立時間 & createdTime & 該路段項目的建立日期與時間 & DateTime \\
  \hline
  更新時間 & updatedTime & 該路段項目最後一次更新的日期與時間 & DateTime \\
  \hline
  關聯識別碼 & relatedId & 用於表示與該路段相關聯的其他項目的識別碼 & String \\
  \hline
\end{longtable}

\subsubsection{軌跡範例}

\subsection{地點 WptItem}

\subsubsection{地點定義與說明}

\begin{longtable}{|c|c|c|c|}
  \hline
  \textbf{欄位名稱} & \textbf{欄位代號} & \textbf{定義} & \textbf{型態} \\
  \hline
  \endfirsthead
  \hline
  \textbf{欄位名稱} & \textbf{欄位代號} & \textbf{定義} & \textbf{型態} \\
  \hline
  \endhead
  
  識別碼 & id & 資料的唯一識別碼 & String \\
  \hline
  名稱 & name & 地點的名稱 & String \\
  \hline
  描述 & description & 地點的詳細說明 & String \\
  \hline
  標籤 & tags & 地點相關的標籤,用於篩選和搜尋 & Set<String> \\
  \hline
  建立時間 & createdTime & 地點建立的時間 & DateTime \\
  \hline
  更新時間 & updatedTime & 地點最後一次被更新的時間 & DateTime \\
  \hline
  緯度 & lat & 地點的緯度 & Double \\
  \hline
  經度 & lon & 地點的經度 & Double \\
  \hline
  海拔 & ele & 地點的海拔 & Double \\
  \hline
  管理識別碼 & relatedId & 使用或管理該點的資源識別碼 & String \\
  \hline
\end{longtable}

\subsubsection{地點範例}

\begin{longtable}{|c|c|c|}
  \hline
  \textbf{欄位名稱} & \textbf{範例 1} & \textbf{範例 2} \\
  \hline
  \endfirsthead
  \hline
  \textbf{欄位名稱} & \textbf{範例 1} & \textbf{範例 2} \\
  \hline
  \endhead
  
  id & WPT001 & WPT002 \\
  \hline
  name & 臺北101 & 西門町 \\
  \hline
  description & 台灣最高的大樓 & 台北的購物和娛樂區 \\
  \hline
  tags & \{"地標", "商業"\} & \{"購物", "娛樂"\} \\
  \hline
  createdTime & 2023-10-10 12:34:56 & 2023-10-13 14:56:00 \\
  \hline
  updatedTime & 2023-10-11 15:30:00 & 2023-10-13 18:20:00 \\
  \hline
  lat & 25.0330 & 25.0418 \\
  \hline
  lon & 121.5654 & 121.5081 \\
  \hline
  ele & 508 & 10 \\
  \hline
  relatedId & TRACK001 & TRACK002 \\
  \hline
\end{longtable}


% 軟體或硬體模組設計
\section{軟體或硬體模組設計}

\newcommand{\customfig}[2]{
  \begin{figure}[H]
    \centering
    \includegraphics[width=0.9\linewidth]{../assets/#1.png}
    \caption{#2}
    \label{#2}
  \end{figure}
}

\customfig{UML-Entity}{Entity 類別關係圖}
\customfig{UML-Repository}{Repository 類別關係圖}
\customfig{UML-DataSource}{DataSource 類別關係圖}

% 軟體或硬體開發環境
\section{軟體或硬體開發環境}

\begin{itemize}
  \item 作業系統環境:Windows, Linux
  \item 主要開發語言:Dart
  \item 程式框架:Flutter
  \item 資料庫:SQLite
  \item 開發工具:Visual Studio Code、Android Studio
  \item 開發環境:Android
  \item 版本控制:Git
\end{itemize}

% 系統測試案例設計
\newcounter{TCcounter}
\setcounter{TCcounter}{0}

\makeatletter
\newcommand{\autoLabelTC}[1]{
  \stepcounter{TCcounter}
  \ifnum\value{TCcounter}<10
    \protected@edef\@currentlabel{TT-TC-0\arabic{TCcounter}}
  \else
    \protected@edef\@currentlabel{TT-TC-\arabic{TCcounter}}
  \fi
  \hspace*{-0.7em}
  \textbf{\@currentlabel}
  \label{#1}
}

\newcommand{\testcaseTable}{}

\newcommand{\addtestcase}[5]{
  \gappto{\testcaseTable}{
    \begin{longtable}{|l|p{13.25cm}|}
      \hline
      \textbf{編號} & \autoLabelTC{#1} \\
      \hline
      \textbf{測試名稱} & #1 \\
      \hline
      % \textbf{測試模組} & #2 \\
      % \hline
      \textbf{相關功能性需求} & #3 \\
      \hline
      \textbf{指令動作} & #4 \\
      \hline
      \textbf{預期結果} & #5 \\
      \hline
    \end{longtable}
  }
}
\makeatother

\section{系統測試案例設計}

\addtestcase
  {軌跡記錄功能測試}
  {}
  {\ref{記錄軌跡}}
  {在應用程式中啟動GPS記錄功能}
  {GPS記錄功能應該成功啟動且軌跡正確地在地圖上反映。}

\addtestcase
  {暫停記錄功能測試}
  {}
  {\ref{記錄軌跡}、\ref{暫停記錄軌跡}}
  {在應用程式中啟動GPS記錄功能。在記錄軌跡的過程中,選擇暫停功能。}
  {軌跡記錄應該停止。}

\addtestcase
  {影像記錄功能測試}
  {}
  {\ref{記錄軌跡}、\ref{從APP開啟相機}}
  {在應用程式中啟動GPS記錄功能。在應用程式內點擊開啟相機的功能。進行拍照或錄影。}
  {成功開啟手機的相機功能,且照片或影片能成功儲存。}

\addtestcase
  {音訊記錄功能測試}
  {}
  {\ref{記錄軌跡}、\ref{從APP錄音}}
  {在應用程式中啟動GPS記錄功能。在應用程式內點擊開啟麥克風的功能。進行錄音。}
  {成功開啟麥克風並進行錄音,且音訊能成功儲存。}

\addtestcase
  {文字記錄功能測試}
  {}
  {\ref{記錄軌跡}、\ref{記錄文字}}
  {在應用程式中啟動GPS記錄功能。在文字記錄功能中,輸入日記或心得。}
  {輸入的文字應該成功儲存。}

\addtestcase
  {快速紀錄文字資料測試}
  {}
  {\ref{記錄軌跡}、\ref{快速記錄}}
  {查看手機通知並於Travel Tracker的持續性通知打開動作選項,選擇快速紀錄後輸入文字,然後點選送出。}
  {輸入的文字應該成功儲存。}

\addtestcase
  {顯示旅程及軌跡測試}
  {}
  {\ref{檢視旅程列表}、\ref{切換各個旅程的可見性}、\ref{檢視軌跡列表}、\ref{地圖上顯示軌跡}}
  {點擊管理列表按鈕,顯示旅程列表。點擊查看圖標按鈕隱藏/取消隱藏旅程,確認至少有一個非隱藏狀態的旅程。點擊旅程詳細資料按鈕,顯示旅程詳細資料。左滑切換至軌跡列表,確認至少有一個軌跡。}
  {旅程中的軌跡正確顯示在地圖上。}

\addtestcase
  {顯示旅遊資料測試(一)}
  {}
  {\ref{篩選旅遊資料}、\ref{地圖旁以時間軸方式顯示旅遊資料}、\ref{軌跡上以群集顯示旅遊資料}}
  {確認至少有一個非隱藏狀態的旅程且至少具有一個以上的軌跡及資料。點選篩選資料按鈕,顯示篩選\&排序介面。根據日期、標籤等條件或是地理位置篩選和搜尋旅遊資料。}
  {軌跡和資料正確顯示在地圖上。}

\addtestcase
  {顯示旅遊資料測試(二)}
  {}
  {\ref{切換至旅遊資料頁面}、\ref{篩選旅遊資料}}
  {確認至少有一個以上的資料。點選切換至旅遊資料頁面按鈕,顯示旅遊資料頁面。點選篩選資料按鈕,顯示篩選\&排序介面。根據日期、標籤等條件或是地理位置篩選和搜尋旅遊資料。}
  {資料正確顯示在旅遊資料頁面。}

\addtestcase
  {創建旅程測試}
  {}
  {\ref{創建新旅程}}
  {點選旅程管理按鈕,點選功能選單按鈕,點選新增旅程圖示,輸入旅程名稱。}
  {成功創建旅程。}

\addtestcase
  {編輯旅程測試}
  {}
  {\ref{編輯旅程的標題}}
  {點擊旅程管理按鈕,顯示旅程列表,長按旅程,跳出選單,選擇編輯旅程標題,編輯旅程標題。}
  {成功編輯旅程。}

\addtestcase
  {編輯軌跡測試}
  {}
  {\ref{檢視軌跡列表}、\ref{編輯軌跡的標題}}
  {點擊旅程詳細資料按鈕,顯示旅程詳細資料,左滑至軌跡列表,長按軌跡標,跳出選單,選擇編輯軌跡標題,編輯軌跡標題。}
  {成功編輯軌跡。}

\addtestcase
  {編輯個別旅遊資料測試}
  {}
  {\ref{切換至旅遊資料頁面}、\ref{檢視個別旅遊資料}、\ref{編輯旅遊資料資訊}}
  {在地圖頁面點選主頁切換按鈕,切換至旅遊資料頁面,點選某個旅遊資料,顯示資料本身與功能選項,選擇編輯標題、標籤、或日期,編輯資料。}
  {資料的訊息被成功更新。}

\addtestcase
  {刪除旅程測試}
  {}
  {\ref{刪除旅程}}
  {點擊旅程管理按鈕,顯示旅程列表,長按旅程,跳出選單,選擇刪除旅程,跳出確認框,點選確認。}
  {成功刪除旅程。}

\addtestcase
  {刪除軌跡測試}
  {}
  {\ref{檢視旅程的統計資料}、\ref{刪除軌跡}}
  {點擊旅程詳細資料按鈕,顯示旅程詳細資料,左滑至軌跡列表,長按軌跡標,跳出選單,選擇刪除軌跡,跳出確認框,點選確認。}
  {成功刪除軌跡。}

\addtestcase
  {刪除旅遊資料測試}
  {}
  {\ref{切換至旅遊資料頁面}、\ref{刪除旅遊資料}}
  {在地圖頁面點選主頁切換按鈕,切換至旅遊資料頁面,點選某個旅遊資料,顯示資料本身與功能選項,選擇刪除資料,跳出確認框,點選確認。}
  {成功刪除旅遊資料。}

\addtestcase
  {地圖控制測試}
  {}
  {\ref{基礎控制動作}, \ref{定位至目前位置}}
  {在地圖上縮放、移動、旋轉地圖或是點選定位按鈕。}
  {地圖正確反應。}

\addtestcase
  {定位至指定軌跡測試}
  {}
  {\ref{檢視軌跡列表}、\ref{定位至指定軌跡}}
  {點擊旅程詳細資料按鈕,顯示旅程詳細資料,左滑至軌跡列表,點選某一軌跡。}
  {地圖正確反應,定位至指定軌跡。}

\addtestcase
  {定位至指定旅遊資料測試}
  {}
  {\ref{定位至指定旅遊資料}}
  {點擊時間軸上的某一個時段。}
  {地圖正確反應,定位至指定旅遊資料。}

\addtestcase
  {匯出軌跡圖測試}
  {}
  {\ref{匯出軌跡圖}}
  {點選旅程管理按鈕,點選功能選單按鈕,點選匯出按鈕,選擇匯出位置。}
  {成功匯出軌跡圖。}

\addtestcase
  {分享測試}
  {}
  {\ref{分享到社群軟體}}
  {在地圖頁面點選主頁切換按鈕,切換至旅遊資料頁面,點選某個旅遊資料,顯示資料本身與功能選項,選擇分享,跳出分享選單,選擇要分享的目標app。}
  {成功分享資料。}

\addtestcase
  {AI 功能測試}
  {}
  {\ref{與AI對話來篩選旅遊資料}、\ref{AI對話編刪旅遊資料}, \ref{AI自動辨識並標記照片}}
  {點選AI助手按鈕,輸入文字訊息。}
  {AI根據使用者輸入的內容正確完成任務。}

\testcaseTable


% 系統測試報告
\section{系統測試報告}

\subsection{測試環境}

\subsubsection{硬體需求}
 
\begin{tabular}{|c|c|c|c|}
  \hline
  \multirow{2}{*}{類型} & \multirow{2}{*}{數量} & \multirow{2}{*}{名稱} & 規格 \\
  \cline{4-4} & & & 型號 \\
  \hline
  \multirow{2}{*}{Android - 手機} & \multirow{2}{*}{2} & 處理器 & Snapdragon 778G 2.4 GHz \\
  \cline{3-4} & & 記憶體 & 6 GB \\
  \hline
  \multirow{2}{*}{Android - 平板} & \multirow{2}{*}{1} & 處理器 & Snapdragon 720G 2.3 GHz \\
  \cline{3-4} & & 記憶體 & 4 GB \\
  \hline
\end{tabular}

\subsubsection{軟體需求}

\begin{tabular}{|c|c|c|c|}
  \hline
  類型 & 數量 & 名稱 & 規格 \\
  \hline
  Android & 2 & 手機 & 作業系統 Android 13.0 \\
  \hline
  Android & 2 & 平板 & 作業系統 Android 13.0 \\
  \hline
\end{tabular}


\subsection{測試結果與分析}

\subsubsection{測試結果}

\newAutoLabel{TT-DT}

\begin{longtable}{|c|c|c|}
  \hline
  \textbf{測試案例編號} & \textbf{測試結果 (pass/fail)} & \textbf{測試案例缺失編號} \\
  \hline
  \endfirsthead
  \hline
  \textbf{測試案例編號} & \textbf{測試結果 (pass/fail)} & \textbf{測試案例缺失編號} \\
  \hline
  \endhead
  \ref{軌跡記錄功能測試} & pass & \\ 
  \hline
  \ref{暫停記錄功能測試} & pass & \\ 
  \hline
  \ref{影像記錄功能測試} & pass & \\ 
  \hline
  \ref{音訊記錄功能測試} & pass & \\ 
  \hline
  \ref{文字記錄功能測試} & pass & \\ 
  \hline
  \ref{快速紀錄文字資料測試} & pass & \autoLabel{快速紀錄文字資料測試缺失報告} \\ 
  \hline
  \ref{顯示旅程及軌跡測試} & pass & \\ 
  \hline
  \ref{顯示旅遊資料測試(一)} & pass & \\ 
  \hline
  \ref{顯示旅遊資料測試(二)} & pass & \\ 
  \hline
  \ref{創建旅程測試} & pass & \\ 
  \hline
  \ref{編輯旅程測試} & pass & \\ 
  \hline
  \ref{編輯軌跡測試} & pass & \\ 
  \hline
  \ref{編輯個別旅遊資料測試} & pass & \\ 
  \hline
  \ref{刪除旅程測試} & pass & \\ 
  \hline
  \ref{刪除軌跡測試} & pass & \\ 
  \hline
  \ref{刪除旅遊資料測試} & pass & \\ 
  \hline
  \ref{地圖控制測試} & pass & \\ 
  \hline
  \rowcolor{orange!25} \ref{定位至指定軌跡測試} & fail & \autoLabel{定位至指定軌跡測試缺失報告} \\ 
  \hline
  \rowcolor{orange!25} \ref{定位至指定旅遊資料測試} & fail & \autoLabel{定位至指定旅遊資料測試缺失報告} \\ 
  \hline
  \ref{匯出軌跡圖測試} & pass & \\ 
  \hline
  \ref{分享測試} & pass & \\ 
  \hline
  \ref{AI 功能測試} & pass & \autoLabel{AI 功能測試缺失報告} \\ 
  \hline
  測試案例透過率 & 90.90\% & \\
  \hline
\end{longtable}

\subsubsection{缺失報告}

\begin{longtable}{|c|c|p{5cm}|c|c|}
  \hline
  \textbf{缺失編號} & \textbf{缺失嚴重性} & \textbf{缺失說明} &  \textbf{測試案例編號} & \textbf{修復狀態} \\
  \hline
  \endfirsthead
  \hline
  \textbf{缺失編號} & \textbf{缺失嚴重性} & \textbf{缺失說明} &  \textbf{測試案例編號} & \textbf{修復狀態} \\
  \hline
  \endhead
  \ref{快速紀錄文字資料測試缺失報告} & 低 & 有時使用者輸入文字後並沒有正確記錄起來。 & \ref{快速紀錄文字資料測試} & Close \\
  \hline
  \rowcolor{orange!25} \ref{定位至指定軌跡測試缺失報告} & 高 & 定位功能無法準確導引至指定軌跡。 & \ref{定位至指定軌跡測試} & Open \\
  \hline
  \rowcolor{orange!25} \ref{定位至指定旅遊資料測試缺失報告} & 高 & 定位功能無法準確找到指定旅遊地點。 & \ref{定位至指定旅遊資料測試} & Open \\
  \hline
  \ref{AI 功能測試缺失報告} & 高 & AI 功能無法正確識別某些情境。 & \ref{AI 功能測試} & Close \\
  \hline
\end{longtable}

\end{document}