\documentclass[12pt]{article}

% 頁面設置
\usepackage[right=20mm, bottom=30mm, left=20mm]{geometry} % 頁面邊界
% \usepackage{multicol} % 頁面分欄
% \setlength{\columnsep}{1pt} % 欄間距
\usepackage{float} % 浮動物件 (圖表 H)

% 字體設定
\usepackage{type1cm} % 設定字體大小
\usepackage{xeCJK} % 中文字體
% \setCJKmainfont{Noto Sans TC}
\setCJKmainfont{kaiu.ttf} % 標楷體

% 套件區
\usepackage{enumitem} % 列表 (enumerate, itemize)
\usepackage{array} % 表格
\usepackage{makecell} % 表格換行
\usepackage{graphicx} % 圖片
\usepackage{longtable} % 長表格
\usepackage{indentfirst} % 自動首行縮排
\usepackage{fancyhdr} % 頁首頁尾
\usepackage{lastpage} % 最後一頁的頁數
\renewcommand{\arraystretch}{1.45} % 表格行高
\usepackage{collcell}

% 使用 hyperref 並設定
\usepackage[unicode=true,pdfusetitle,
 bookmarks=true,bookmarksnumbered=false,bookmarksopen=false,
 breaklinks=false,pdfborder={0 0 0},backref=true,colorlinks=false]
 {hyperref}

% \usepackage{amssymb} % 數學符號
% \usepackage[fleqn]{amsmath} % 數學排版, fleqn 選項讓數學式靠左對齊
% \usepackage{tikz} % 繪圖
% \usepackage{pgfplots} % 圖表
% \usepackage{caption} % 圖表標題
% \usepackage{subcaption} % 圖表子標題
% \usepackage{subfig} % 圖表子圖

% tikz 設定
% \tikzset{every state, accepting/.style={double distance=2pt}}
% \captionsetup[figure]{labelfont={bf},name={圖},labelsep=period}
% \usetikzlibrary{automata, positioning, arrows}

% fancyhdr 設定
\pagestyle{fancy}
\renewcommand{\footnotesize}{\normalsize} % 設定腳註字型大小
\renewcommand{\headrulewidth}{0pt} % 不要有頁首橫線
\renewcommand{\footrulewidth}{0pt} % 不要有頁尾橫線

\lhead{}
\chead{2023年全國大專校院智慧創新暨跨域整合創作競賽 - 作品測試文件}
\rhead{}
\lfoot{}
\cfoot{}
\rfoot{ 共 \pageref{LastPage} 頁 第 \thepage 頁} 

\begin{document}

% \makeatletter % 設定目錄字型大小
% \tableofcontents % 目錄

% 標題
% \title{}
% \author{}
% \date{}
% \maketitle

% 系統目的與範圍、提供服務與技術
\input{info/info.tex}

% 非功能性需求
\input{nfrs/nfrs.tex}

% 功能性需求
\section{系統功能需求}

\begin{itemize}
  \item 旅程:包含一次旅遊的所有軌跡、旅遊資料
  \item 軌跡:使用者用GPS記錄的路徑
  \item 旅遊資料:照片、影片、錄音、文字等多媒體
\end{itemize}

%記錄功能 RC
\newAutoLabel{TT-RC}

\subsection{記錄功能 RC}

\begin{longtable}{|c|p{4.3cm}|p{8.9cm}|}
  \hline
  \textbf{功能需求編號} & \textbf{功能名稱} & \textbf{功能需求描述} \\
  \hline
  \endfirsthead
  \hline
  \textbf{功能需求編號} & \textbf{功能名稱} & \textbf{功能需求描述} \\
  \hline
  \endhead
  \autoLabel{記錄軌跡} & 記錄軌跡 & 使用者可以通過GPS記錄旅程軌跡 \\
  \hline
  \autoLabel{暫停記錄軌跡} & 暫停記錄軌跡 & 允許使用者暫停記錄旅程軌跡 \\
  \hline
  \autoLabel{從APP開啟相機} & 從APP開啟相機 & 使用者可以直接從應用程式開啟相機進行拍照 \\
  \hline
  \autoLabel{從APP錄音} & 從APP錄音 & 使用者可以直接從應用程式開啟麥克風進行錄音 \\
  \hline
  \autoLabel{記錄文字} & 記錄文字 & 允許使用者記錄文字內容,如日記或心得 \\
  \hline
  \autoLabel{快速記錄} & 快速記錄 & 使用者可以通過桌面小工具或通知列快速記錄文字、錄音等等 \\
  \hline
  % TODO
  \autoLabel{自動抓取媒體資料} & 自動抓取媒體資料 & 自動從手機的相簿或其他媒體資料夾中抓取相關旅遊媒體資料 \\
  \hline
\end{longtable}

%檢視軌跡與旅遊資料 RV
\newAutoLabel{TT-RV}

\subsection{檢視軌跡與旅遊資料 RV}

\begin{longtable}{|c|p{4.3cm}|p{8.9cm}|}
  \hline
  \textbf{功能需求編號} & \textbf{功能名稱} & \textbf{功能需求描述} \\
  \hline
  \endfirsthead
  \hline
  \textbf{功能需求編號} & \textbf{功能名稱} & \textbf{功能需求描述} \\
  \hline
  \endhead
  \autoLabel{檢視旅程列表} & 檢視旅程列表 & 展示所有旅程的列表 \\
  \hline
  \autoLabel{切換各個旅程的可見性} & 切換各個旅程的可見性 & 允許使用者切換各個旅程的可見性,可以有多個旅程為可見狀態顯示在地圖上 \\
  \hline
  \autoLabel{隱藏/取消隱藏所有旅程} & 隱藏/取消隱藏所有旅程 & 允許使用者一次隱藏/取消隱藏所有旅程 \\
  \hline
  \autoLabel{檢視旅程的統計資料} & 檢視旅程的統計資料 & 檢視總距離、時間、速度等統計資料 \\
  \hline
  \autoLabel{地圖上顯示軌跡} & 地圖上顯示軌跡 & 在地圖上展示使用者的旅程軌跡 \\
  \hline
  \autoLabel{檢視軌跡列表} & 檢視軌跡列表 & 展示所有可視的旅程的軌跡列表 \\
  \hline
  \autoLabel{切換至旅遊資料頁面} & 切換至旅遊資料頁面 & 使用者能切換到旅遊資料頁面檢視旅遊資料 \\
  \hline
  \autoLabel{選取多個旅遊資料} & 選取多個旅遊資料 & 允許使用者選取多個旅遊資料 \\
  \hline
  \autoLabel{檢視個別旅遊資料} & 檢視個別旅遊資料 & 查看單一旅遊資料的詳細資訊 \\
  \hline
  \autoLabel{篩選旅遊資料} & 篩選旅遊資料 & 根據日期、標籤等條件或是地理位置篩選和搜尋旅遊資料 \\
  \hline
  \autoLabel{地圖旁以時間軸方式顯示旅遊資料} & 地圖旁以時間軸方式顯示旅遊資料 & 旅遊資料以時間軸形式展示,方便查看 \\
  \hline
  \autoLabel{軌跡上以群集顯示旅遊資料} & 軌跡上以群集顯示旅遊資料 & 軌跡上以群集顯示旅遊相關資料,如照片、影片、文字等 \\
  \hline
  \autoLabel{與AI對話來篩選旅遊資料} & 與AI對話來篩選旅遊資料 & 通過AI對話界面協助篩選旅遊資料 \\
  \hline
\end{longtable}


%編輯軌跡與旅遊資料 WR
\newcounter{WRcounter}
\setcounter{WRcounter}{0}

\makeatletter
\newcommand{\autolabelWR}[1]{
  \stepcounter{WRcounter}
  \ifnum\value{WRcounter}<10
    \protected@edef\@currentlabel{TT-WR-0\arabic{WRcounter}}
  \else
    \protected@edef\@currentlabel{TT-WR-\arabic{WRcounter}}
  \fi
  \hspace*{-0.7em}
  \@currentlabel
  \label{#1}
}
\makeatother

\subsection{編輯軌跡與旅遊資料 WR}

\begin{longtable}{|c|p{4.3cm}|p{8.9cm}|}
  \hline
  \textbf{功能需求編號} & \textbf{功能名稱} & \textbf{功能需求描述} \\
  \hline
  \endfirsthead
  \hline
  \textbf{功能需求編號} & \textbf{功能名稱} & \textbf{功能需求描述} \\
  \hline
  \endhead
  % \autolabelWR{軌跡平滑化} & 軌跡平滑化 & 將記錄的軌跡進行平滑處理,使之更符合實際路線 \\
  % \hline
  % \autolabelWR{拉伸軌跡} & 拉伸軌跡 & 允許使用者手動調整軌跡的形狀 \\
  % \hline
  % \autolabelWR{偵測軌跡停留位置並簡化} & 偵測軌跡停留位置並簡化 & 能自動識別軌跡中的停留位置,並將其 GPS 誤差造成的一團路徑簡化為一個點 \\
  % \hline
  \autolabelWR{編輯旅程的標題} & 編輯旅程的標題 & 允許使用者為旅程編輯標題 \\
  \hline
  \autolabelWR{編輯軌跡的標題} & 編輯軌跡的標題 & 允許使用者為軌跡編輯標題 \\
  \hline
  \autolabelWR{編輯旅遊資料的標題} & 編輯旅遊資料的標題 & 允許使用者為旅遊資料編輯標題 \\
  \hline
  \autolabelWR{編輯旅遊資料的標籤} & 編輯旅遊資料的標籤 & 允許使用者為旅遊資料添加或修改標籤 \\
  \hline
  \autolabelWR{編輯旅遊資料的日期} & 編輯旅遊資料的日期 & 允許使用者為旅遊資料添加或修改日期 \\
  \hline
  \autolabelWR{刪除旅程} & 刪除旅程 & 允許使用者刪除指定的旅程 \\
  \hline
  \autolabelWR{刪除軌跡} & 刪除軌跡 & 允許使用者刪除指定的軌跡 \\
  \hline
  \autolabelWR{刪除旅遊資料} & 刪除旅遊資料 & 允許使用者刪除指定的旅遊資料 \\
  \hline
  % \autolabelWR{批量處理旅遊資料} & 批量處理旅遊資料 & 選取多個旅遊資料並一起編輯或刪除 \\
  % \hline
  \autolabelWR{與AI對話來選取並編輯與刪除旅遊資料} & 與AI對話來選取並編輯與刪除旅遊資料 & 通過AI對話協助選取、編輯或刪除旅遊資料 \\
  \hline
  \autolabelWR{AI自動辨識並標記照片} & AI自動辨識並標記照片 & AI會自動分析並標記照片中的物體或景物 \\
  \hline
\end{longtable}

% 地圖視圖控制 MC
\newcounter{MCcounter}
\setcounter{MCcounter}{0}

\makeatletter
\newcommand{\autolabelMC}[1]{
  \stepcounter{MCcounter}
  \ifnum\value{MCcounter}<10
    \protected@edef\@currentlabel{TT-F-MC-0\arabic{MCcounter}}
  \else
    \protected@edef\@currentlabel{TT-F-MC-\arabic{MCcounter}}
  \fi
  \hspace*{-0.7em}
  \@currentlabel
  \label{#1}
}
\makeatother

\subsection{地圖視圖控制 MC}

\begin{longtable}{|c|p{4.3cm}|p{8.9cm}|}
  \hline
  \textbf{功能需求編號} & \textbf{功能名稱} & \textbf{功能需求描述} \\
  \hline
  \endfirsthead
  \hline
  \textbf{功能需求編號} & \textbf{功能名稱} & \textbf{功能需求描述} \\
  \hline
  \endhead
  \autolabelMC{基礎控制動作} & 基礎控制動作 & 使用者可以縮放、移動、旋轉地圖 \\
  \hline
  \autolabelMC{定位至目前位置} & 定位至目前位置 & 讓使用者能快速定位到當前位置 \\
  \hline
  \autolabelMC{定位至指定軌跡} & 定位至指定軌跡 & 允許使用者快速定位至特定軌跡 \\
  \hline
  \autolabelMC{定位至指定旅遊資料} & 定位至指定旅遊資料 & 允許使用者快速定位至特定旅遊資料 \\
  \hline
  \autolabelMC{地圖樣式選擇} & 地圖樣式選擇 & 使用者可以選擇不同的地圖主題和樣式,如魯地圖、OpenStreetMap、夜間模式、衛星圖像等 \\
  \hline
\end{longtable}

% 資料匯入、匯出 EP
\newcounter{EPcounter}
\setcounter{EPcounter}{0}

\makeatletter
\newcommand{\autolabelEP}[1]{
  \stepcounter{EPcounter}
  \ifnum\value{EPcounter}<10
    \protected@edef\@currentlabel{TT-F-EP-0\arabic{EPcounter}}
  \else
    \protected@edef\@currentlabel{TT-F-EP-\arabic{EPcounter}}
  \fi
  \hspace*{-0.7em}
  \@currentlabel
  \label{#1}
}
\makeatother

\subsection{資料匯入、匯出 EP}

\begin{longtable}{|c|p{4.3cm}|p{8.9cm}|}
  \hline
  \textbf{功能需求編號} & \textbf{功能名稱} & \textbf{功能需求描述} \\
  \hline
  \endfirsthead
  \hline
  \textbf{功能需求編號} & \textbf{功能名稱} & \textbf{功能需求描述} \\
  \hline
  \endhead
  \autolabelEP{匯出旅程、軌跡、旅遊資料} & 匯出旅程、軌跡、旅遊資料 & 允許使用者將各式資料匯出為特定格式的檔案 \\
  \hline
  \autolabelEP{匯入旅程、軌跡、旅遊資料} & 匯入旅程、軌跡、旅遊資料 & 允許使用者將各式資料匯入,並且能根據軌跡的時間戳記從手機內自動讀取相應的媒體 \\
  \hline
  \autolabelEP{匯出軌跡圖} & 匯出軌跡圖 & 允許使用者產生整個軌跡在地圖上的截圖 \\
  \hline
  \autolabelEP{分享到社群軟體} & 分享到社群軟體 & 允許使用者選擇照片,並將照片分享到社群軟體 \\
  \hline
\end{longtable}

% 旅程管理 TM
\newcounter{TMcounter}
\setcounter{TMcounter}{0}

\makeatletter
\newcommand{\autolabelTM}[1]{
  \stepcounter{TMcounter}
  \ifnum\value{TMcounter}<10
    \protected@edef\@currentlabel{TT-TM-0\arabic{TMcounter}}
  \else
    \protected@edef\@currentlabel{TT-TM-\arabic{TMcounter}}
  \fi
  \hspace*{-0.7em}
  \@currentlabel
  \label{#1}
}
\makeatother

\subsection{旅程管理 TM}
\begin{longtable}{|c|p{4.3cm}|p{8.9cm}|}
  \hline
  \textbf{功能需求編號} & \textbf{功能名稱} & \textbf{功能需求描述} \\
  \hline
  \endfirsthead
  \hline
  \textbf{功能需求編號} & \textbf{功能名稱} & \textbf{功能需求描述} \\
  \hline
  \endhead
  \autolabelTM{創建新旅程} & 創建新旅程 & 允許使用者創建新的旅程 \\
  \hline
  \autolabelTM{切換目前記錄旅程} & 切換目前記錄旅程\footnote[1] & 允許使用者切換正在記錄的旅程 \\
  \hline
  \autolabelTM{切換目前顯示旅程} & 切換目前顯示旅程 & 允許使用者選擇目前正在顯示的旅程以方便查看 \\
  \hline
  \autolabelTM{一鍵回到目前旅程} & 一鍵回到目前旅程 & 允許使用者快速將當前記錄中的旅程做為顯示中的旅程 \\
  \hline
  \autolabelTM{旅程重新命名} & 旅程重新命名 & 允許使用者對旅程進行重新命名 \\
  \hline
  \autolabelTM{顯示旅程總距離、時間等等} & 顯示旅程總距離、時間等等 & 提供使用者查看旅程的總距離和時間等資訊,以便了解整體旅程的概況 \\
  \hline
\end{longtable}

\footnotetext[1]{顯示中與記錄中的旅程是獨立的,為了讓使用者可以在記錄B旅遊時回去看A旅遊的資料}

  

% 使用案例
\newcounter{UCcounter}
\setcounter{UCcounter}{0}

\makeatletter
\newcommand{\autolabelUC}[1]{
  \stepcounter{UCcounter}
  \ifnum\value{UCcounter}<10
    \protected@edef\@currentlabel{TT-UC-0\arabic{UCcounter}}
  \else
    \protected@edef\@currentlabel{TT-UC-\arabic{UCcounter}}
  \fi
  \hspace*{-0.7em}
  \textbf{\@currentlabel}
  \label{#1}
}
\makeatother

\section{一般性的系統功能操作使用案例之劇本(Scenario)}

\begin{longtable}{|l|p{13.25cm}|}
  \hline
  \textbf{使用者案例} & \autolabelUC{軌跡和資料記錄} \\
  \hline
  \textbf{案例名稱} & 軌跡和資料記錄 \\
  \hline
  \textbf{相關功能性需求} & \ref{記錄軌跡}、\ref{暫停記錄軌跡}、\ref{從APP開啟相機}、\ref{從APP錄音}、\ref{記錄文字}、\newline \ref{快速記錄} \\
  \hline
  \textbf{使用者} & 一般使用者 \\
  \hline
  \textbf{前置條件} & 使用者已安裝相應的應用程式並準備記錄旅遊資料 \\
  \hline
  \textbf{說明} & 使用者可以在旅途中使用手機上應用程式記錄軌跡、拍照、錄音或文字,並可以使用快捷功能更快速的記錄資料,也可以隨時暫停記錄。 \\
  \hline
  \textbf{使用者操作} & 
  1. 使用者開啟記錄功能開始記錄軌跡,並有暫停鍵或結束鍵以提供暫停或結束記錄。\newline
  2. 使用者可以從APP中開啟手機上相機功能進行拍照或錄影。\newline
  3. 使用者可以從APP中開啟錄音功能開始錄音。\newline
  4. 使用者可以在APP中開啟文字記錄功能開始記錄文字。\newline
  5. 使用者可透過APP提供的桌面小工具或通知列選項,讓使用者可以不用打開APP就可以直接快速地記錄旅遊資料。 \\
  \hline
  \textbf{例外處理} & 若使用者的裝置沒有授予相關權限(例如鏡頭、錄音),則應提供適當的提示訊息。 \\
  \hline
  \textbf{測試方案} & 測試在沒有權限的情況下開啟相關功能,確認有提供提示訊息。 \\
  \hline
\end{longtable}

\begin{longtable}{|l|p{13.25cm}|}
  \hline
  \textbf{使用者案例} & \autolabelUC{檢視旅遊資料} \\
  \hline
  \textbf{案例名稱} & 檢視旅遊資料 \\
  \hline
  \textbf{相關功能性需求} & \ref{軌跡上顯示旅遊資料}、\ref{地圖旁以時間軸方式顯示旅遊資料}、\ref{切換圖庫顯示旅遊資料頁面}、\ref{檢視個別旅遊資料} \\
  \hline
  \textbf{使用者} & 一般使用者 \\
  \hline
  \textbf{前置條件} & 需將旅遊資料記錄好,包含行進軌跡、影像、錄音或記錄之文字。 \\
  \hline
  \textbf{說明} & 使用者可以依自己想瀏覽旅遊資料的形式選擇合適的旅遊資料瀏覽方式,瀏覽方式有:在地圖軌跡上顯示旅遊資料、在地圖旁以時間軸方式顯示對應軌跡的旅遊資料、格狀式顯示旅遊資料或條列式顯示旅遊資料,此外,也可以檢視個別或單一影像、錄音或文字。 \\
  \hline
  \textbf{描述(操作流程)} & 
  1. 使用者可直接在地圖頁籤縮放地圖來查看軌跡上的影像。\newline
  2. 使用者在地圖頁籤,可在地圖旁看到時間軸,以上下滑動的方式瀏覽旅遊資料,並可對應到地圖上的軌跡點。\newline
  3. 使用者可以向左滑,或是點擊資料頁籤,將時間軸上的資料展開檢視。\newline
  4. 使用者在資料頁籤時,從檢視模式選項選擇以格狀方式顯示旅遊資料。\newline
  5. 使用者在資料頁籤時,從檢視模式選項選擇以條列方式顯示旅遊資料。\newline
  6. 使用者可在任何模式點選個別旅遊資料並瀏覽。 \\
  \hline
  \textbf{例外處理} & 若使用者未記錄行進軌跡、影像、錄音或文字,則系統會顯示「無資料顯示」。 \\
  \hline
  \textbf{測試方案} & 使用者在未新增旅遊資料並點選瀏覽紀錄之功能時,確認系統是否顯示「無資料顯示」。 \\
  \hline
\end{longtable}

% \begin{longtable}{|l|p{13.25cm}|}
%   \hline
%   \textbf{使用者案例} & \textbf{TT-UC-03} \\
%   \hline
%   \textbf{案例名稱} & 軌跡管理 \\
%   \hline
%   \textbf{相關功能性需求} & 軌跡平滑化、拉伸軌跡、偵測軌跡停留位置並簡化 \\
%   \hline
%   \textbf{使用者} & 一般使用者 \\
%   \hline
%   \textbf{前置條件} & 使用者已記錄完軌跡。 \\
%   \hline
%   \textbf{說明} & 使用者可以對已經記錄的軌跡進行平滑化、拉伸或是偵測軌跡停留位置並簡化。 \\
%   \hline
%   \textbf{描述(操作流程)} & 
%   1. 使用者選擇已記錄的軌跡。\newline
%   2. 使用者選擇平滑化軌跡。\newline
%   3. 使用者選擇拉伸軌跡。\newline
%   4. 若平滑化軌跡後軌跡有雜亂的點,使用者可以選擇偵測軌跡停留位置並將雜亂的軌跡路線簡化成一個點。 \\
%   \hline
%   \textbf{例外處理} & 使用者沒有記錄軌跡,則系統應提示「無軌跡」。 \\
%   \hline
%   \textbf{測試方案} & 測試不輸入軌跡,確保所有修改功能對使用者顯示對應之提示。 \\
%   \hline
% \end{longtable}

\begin{longtable}{|l|p{13.25cm}|}
  \hline
  \textbf{使用者案例} & \autolabelUC{旅遊資料管理} \\
  \hline
  \textbf{案例名稱} & 旅遊資料管理 \\
  \hline
  \textbf{相關功能性需求} & \ref{編輯旅遊資料的標籤}、\ref{編輯旅遊資料的日期}、\ref{刪除旅遊資料}、\ref{與AI對話來選取並編輯與刪除旅遊資料}、\ref{AI自動辨識並標記照片} \\
  \hline
  \textbf{使用者} & 一般使用者 \\
  \hline
  \textbf{前置條件} & 使用者已記錄旅遊資料(影像、錄音或文字) \\
  \hline
  \textbf{說明} & 使用者可以編輯標籤、日期,刪除資料,或批量處理。此外,還可以透過AI的協助進行更智能的資料管理,例如自動辨識標記照片並在搜尋欄中搜尋已標記的照片。 \\
  \hline
  \textbf{使用者操作} & 
  1. 使用者選擇旅遊資料。\newline
  2. 使用者編輯標籤和日期。\newline
  3. 使用者刪除旅遊資料。\newline
  4. 使用者與AI對話來選取、編輯或刪除資料。\newline
  5. AI自動辨識並標記照片。\newline
  6. 使用者在搜尋欄中搜尋已標記的照片。 \\
  \hline
  \textbf{例外處理} & 若資料缺失、系統無法正確解析或AI無法辨別標記照片,則應提供適當的錯誤訊息。 \\
  \hline
  \textbf{測試方案} & 
  1. 使用者在未新增旅遊資料時,確認系統是否顯示「無資料」。\newline
  2. 測試將照片放置系統中,AI是否可以自動辨認並標記照片。\newline
  3. 測試在搜尋欄中搜尋已標記的內容,確認系統是否可顯示相對應的內容。 \\
  \hline
\end{longtable}

% 系統架構設計

% 系統介面設計

% 軟體或硬體架構設計

% 軟體或硬體模組設計

% 軟體或硬體開發環境

% 系統測試案例設計

% 系統測試報告

\end{document}