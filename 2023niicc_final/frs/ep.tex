\newcounter{EPcounter}
\setcounter{EPcounter}{0}

\makeatletter
\newcommand{\autolabelEP}[1]{
  \stepcounter{EPcounter}
  \ifnum\value{EPcounter}<10
    \protected@edef\@currentlabel{TT-F-EP-0\arabic{EPcounter}}
  \else
    \protected@edef\@currentlabel{TT-F-EP-\arabic{EPcounter}}
  \fi
  \hspace*{-0.7em}
  \@currentlabel
  \label{#1}
}
\makeatother

\subsection{資料匯入、匯出 EP}

\begin{longtable}{|c|p{4.3cm}|p{8.9cm}|}
  \hline
  \textbf{功能需求編號} & \textbf{功能名稱} & \textbf{功能需求描述} \\
  \hline
  \endfirsthead
  \hline
  \textbf{功能需求編號} & \textbf{功能名稱} & \textbf{功能需求描述} \\
  \hline
  \endhead
  \autolabelEP{匯出旅程、軌跡、旅遊資料} & 匯出旅程、軌跡、旅遊資料 & 允許使用者將各式資料匯出為特定格式的檔案 \\
  \hline
  \autolabelEP{匯入旅程、軌跡、旅遊資料} & 匯入旅程、軌跡、旅遊資料 & 允許使用者將各式資料匯入,並且能根據軌跡的時間戳記從手機內自動讀取相應的媒體 \\
  \hline
  \autolabelEP{匯出軌跡圖} & 匯出軌跡圖 & 允許使用者產生整個軌跡在地圖上的截圖 \\
  \hline
  \autolabelEP{分享到社群軟體} & 分享到社群軟體 & 允許使用者選擇照片,並將照片分享到社群軟體 \\
  \hline
\end{longtable}