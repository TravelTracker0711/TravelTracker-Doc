\newcounter{TMcounter}
\setcounter{TMcounter}{0}

\makeatletter
\newcommand{\autolabelTM}[1]{
  \stepcounter{TMcounter}
  \ifnum\value{TMcounter}<10
    \protected@edef\@currentlabel{TT-TM-0\arabic{TMcounter}}
  \else
    \protected@edef\@currentlabel{TT-TM-\arabic{TMcounter}}
  \fi
  \hspace*{-0.7em}
  \@currentlabel
  \label{#1}
}
\makeatother

\subsection{旅程管理 TM}
\begin{longtable}{|c|p{4.3cm}|p{8.9cm}|}
  \hline
  \textbf{功能需求編號} & \textbf{功能名稱} & \textbf{功能需求描述} \\
  \hline
  \endfirsthead
  \hline
  \textbf{功能需求編號} & \textbf{功能名稱} & \textbf{功能需求描述} \\
  \hline
  \endhead
  \autolabelTM{創建新旅程} & 創建新旅程 & 允許使用者創建新的旅程 \\
  \hline
  \autolabelTM{切換目前記錄旅程} & 切換目前記錄旅程\footnote[1] & 允許使用者切換正在記錄的旅程 \\
  \hline
  \autolabelTM{切換各個旅程的可見性} & 切換各個旅程的可見性 & 允許使用者切換各個旅程的可見性以方便查看 \\
  \hline
\end{longtable}

\footnotetext[1]{顯示中與記錄中的旅程是獨立的,為了讓使用者可以在記錄B旅遊時回去看A旅遊的資料}

  