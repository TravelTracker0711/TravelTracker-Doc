\newcounter{WRcounter}
\setcounter{WRcounter}{0}

\makeatletter
\newcommand{\autolabelWR}[1]{
  \stepcounter{WRcounter}
  \ifnum\value{WRcounter}<10
    \protected@edef\@currentlabel{TT-WR-0\arabic{WRcounter}}
  \else
    \protected@edef\@currentlabel{TT-WR-\arabic{WRcounter}}
  \fi
  \hspace*{-0.7em}
  \@currentlabel
  \label{#1}
}
\makeatother

\subsection{編輯軌跡與旅遊資料 WR}

\begin{longtable}{|c|p{4.3cm}|p{8.9cm}|}
  \hline
  \textbf{功能需求編號} & \textbf{功能名稱} & \textbf{功能需求描述} \\
  \hline
  \endfirsthead
  \hline
  \textbf{功能需求編號} & \textbf{功能名稱} & \textbf{功能需求描述} \\
  \hline
  \endhead
  \autolabelWR{軌跡平滑化} & 軌跡平滑化 & 將記錄的軌跡進行平滑處理,使之更符合實際路線 \\
  \hline
  \autolabelWR{拉伸軌跡} & 拉伸軌跡 & 允許使用者手動調整軌跡的形狀 \\
  \hline
  \autolabelWR{偵測軌跡停留位置並簡化} & 偵測軌跡停留位置並簡化 & 能自動識別軌跡中的停留位置,並將其 GPS 誤差造成的一團路徑簡化為一個點 \\
  \hline
  \autolabelWR{編輯旅遊資料的標籤} & 編輯旅遊資料的標籤 & 允許使用者為旅遊資料添加或修改標籤 \\
  \hline
  \autolabelWR{編輯旅遊資料的日期} & 編輯旅遊資料的日期 & 允許使用者為旅遊資料添加或修改日期 \\
  \hline
  \autolabelWR{刪除旅遊資料} & 刪除旅遊資料 & 允許使用者刪除指定的旅遊資料 \\
  \hline
  \autolabelWR{批量處理旅遊資料} & 批量處理旅遊資料 & 選取多個旅遊資料並一起編輯或刪除 \\
  \hline
  \autolabelWR{與AI對話來選取並編輯與刪除旅遊資料} & 與AI對話來選取並編輯與刪除旅遊資料 & 通過AI對話協助選取、編輯或刪除旅遊資料 \\
  \hline
  \autolabelWR{AI自動辨識並標記照片} & AI自動辨識並標記照片 & AI會自動分析並標記照片中的物體或景物 \\
  \hline
\end{longtable}