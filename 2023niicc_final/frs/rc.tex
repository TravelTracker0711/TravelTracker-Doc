\newcounter{RCcounter}
\setcounter{RCcounter}{0}

\makeatletter
\newcommand{\autolabelRC}[1]{
  \stepcounter{RCcounter}
  \ifnum\value{RCcounter}<10
    \protected@edef\@currentlabel{TT-RC-0\arabic{RCcounter}}
  \else
    \protected@edef\@currentlabel{TT-RC-\arabic{RCcounter}}
  \fi
  \hspace*{-0.7em}
  \@currentlabel
  \label{#1}
}
\makeatother

\subsection{記錄功能 RC}

\begin{longtable}{|c|p{4.3cm}|p{8.9cm}|}
  \hline
  \textbf{功能需求編號} & \textbf{功能名稱} & \textbf{功能需求描述} \\
  \hline
  \endfirsthead
  \hline
  \textbf{功能需求編號} & \textbf{功能名稱} & \textbf{功能需求描述} \\
  \hline
  \endhead
  \autolabelRC{記錄軌跡} & 記錄軌跡 & 使用者可以通過GPS記錄旅程軌跡 \\
  \hline
  \autolabelRC{暫停記錄軌跡} & 暫停記錄軌跡 & 允許使用者暫停記錄旅程軌跡 \\
  \hline
  \autolabelRC{從APP開啟相機} & 從APP開啟相機 & 使用者可以直接從應用程式開啟相機進行拍照 \\
  \hline
  \autolabelRC{從APP錄音} & 從APP錄音 & 使用者可以直接從應用程式開啟麥克風進行錄音 \\
  \hline
  \autolabelRC{記錄文字} & 記錄文字 & 允許使用者記錄文字內容,如日記或心得 \\
  \hline
  \autolabelRC{快速記錄} & 快速記錄 & 使用者可以通過桌面小工具或通知列快速記錄文字、錄音等等 \\
  \hline
  \autolabelRC{自動抓取媒體資料} & 自動抓取媒體資料 & 自動從手機的相簿或其他媒體資料夾中抓取相關旅遊媒體資料 \\
  \hline
\end{longtable}