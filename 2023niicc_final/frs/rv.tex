\newcounter{RVcounter}
\setcounter{RVcounter}{0}

\makeatletter
\newcommand{\autolabelRV}[1]{
  \stepcounter{RVcounter}
  \ifnum\value{RVcounter}<10
    \protected@edef\@currentlabel{TT-RV-0\arabic{RVcounter}}
  \else
    \protected@edef\@currentlabel{TT-RV-\arabic{RVcounter}}
  \fi
  \hspace*{-0.7em}
  \@currentlabel
  \label{#1}
}
\makeatother

\subsection{檢視軌跡與旅遊資料 RV}

\begin{longtable}{|c|p{4.3cm}|p{8.9cm}|}
  \hline
  \textbf{功能需求編號} & \textbf{功能名稱} & \textbf{功能需求描述} \\
  \hline
  \endfirsthead
  \hline
  \textbf{功能需求編號} & \textbf{功能名稱} & \textbf{功能需求描述} \\
  \hline
  \endhead
  \autolabelRV{地圖上顯示軌跡} & 地圖上顯示軌跡 & 在地圖上展示使用者的旅程軌跡 \\
  \hline
  \autolabelRV{軌跡上顯示旅遊資料} & 軌跡上顯示旅遊資料 & 軌跡上顯示旅遊相關資料,如照片、影片、文字等 \\
  \hline
  \autolabelRV{地圖旁以時間軸方式顯示旅遊資料} & 地圖旁以時間軸方式顯示旅遊資料 & 旅遊資料以時間軸形式展示,方便查看 \\
  \hline
  \autolabelRV{格狀顯示旅遊資料} & 格狀顯示旅遊資料 & 使用者能切換到格狀頁面檢視旅遊資料 \\
  \hline
  \autolabelRV{條列式詳細顯示旅遊資料} & 條列式詳細顯示旅遊資料 & 使用者能切換到條列式頁面,以便於檢視旅遊資料的詳細內容 \\
  \hline
  \autolabelRV{檢視個別旅遊資料} & 檢視個別旅遊資料 & 查看單一旅遊資料的詳細資訊 \\
  \hline
  \autolabelRV{利用標籤、日期等項目篩選旅遊資料} & 利用標籤、日期等項目篩選旅遊資料 & 根據日期、標籤等條件篩選和搜尋旅遊資料 \\
  \hline
  \autolabelRV{用地理位置篩選旅遊資料} & 用地理位置篩選旅遊資料 & 一次選取某個位置周圍的旅遊資料 \\
  \hline
  \autolabelRV{與AI對話來篩選旅遊資料} & 與AI對話來篩選旅遊資料 & 通過AI對話界面協助篩選旅遊資料 \\
  \hline
\end{longtable}