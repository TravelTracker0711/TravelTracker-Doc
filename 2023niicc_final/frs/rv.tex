\newAutoLabel{TT-RV}

\subsection{檢視軌跡與旅遊資料 RV}

\begin{longtable}{|c|p{4.3cm}|p{8.9cm}|}
  \hline
  \textbf{功能需求編號} & \textbf{功能名稱} & \textbf{功能需求描述} \\
  \hline
  \endfirsthead
  \hline
  \textbf{功能需求編號} & \textbf{功能名稱} & \textbf{功能需求描述} \\
  \hline
  \endhead
  \autoLabel{檢視旅程列表} & 檢視旅程列表 & 展示所有旅程的列表 \\
  \hline
  \autoLabel{切換各個旅程的可見性} & 切換各個旅程的可見性 & 允許使用者切換各個旅程的可見性,可以有多個旅程為可見狀態顯示在地圖上 \\
  \hline
  \autoLabel{隱藏/取消隱藏所有旅程} & 隱藏/取消隱藏所有旅程 & 允許使用者一次隱藏/取消隱藏所有旅程 \\
  \hline
  \autoLabel{檢視旅程的統計資料} & 檢視旅程的統計資料 & 檢視總距離、時間、速度等統計資料 \\
  \hline
  \autoLabel{地圖上顯示軌跡} & 地圖上顯示軌跡 & 在地圖上展示使用者的旅程軌跡 \\
  \hline
  \autoLabel{檢視軌跡列表} & 檢視軌跡列表 & 展示所有可視的旅程的軌跡列表 \\
  \hline
  \autoLabel{切換至旅遊資料頁面} & 切換至旅遊資料頁面 & 使用者能切換到旅遊資料頁面檢視旅遊資料 \\
  \hline
  \autoLabel{選取多個旅遊資料} & 選取多個旅遊資料 & 允許使用者選取多個旅遊資料 \\
  \hline
  \autoLabel{檢視個別旅遊資料} & 檢視個別旅遊資料 & 查看單一旅遊資料的詳細資訊 \\
  \hline
  \autoLabel{篩選旅遊資料} & 篩選旅遊資料 & 根據日期、標籤等條件或是地理位置篩選和搜尋旅遊資料 \\
  \hline
  \autoLabel{地圖旁以時間軸方式顯示旅遊資料} & 地圖旁以時間軸方式顯示旅遊資料 & 旅遊資料以時間軸形式展示,方便查看 \\
  \hline
  \autoLabel{軌跡上以群集顯示旅遊資料} & 軌跡上以群集顯示旅遊資料 & 軌跡上以群集顯示旅遊相關資料,如照片、影片、文字等 \\
  \hline
  \autoLabel{與AI對話來篩選旅遊資料} & 與AI對話來篩選旅遊資料 & 通過AI對話界面協助篩選旅遊資料 \\
  \hline
\end{longtable}
