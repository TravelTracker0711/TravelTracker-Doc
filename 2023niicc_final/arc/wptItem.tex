\subsection{地點 WptItem}

\subsubsection{地點定義與說明}

\begin{longtable}{|c|c|c|c|}
  \hline
  \textbf{欄位名稱} & \textbf{欄位代號} & \textbf{定義} & \textbf{型態} \\
  \hline
  \endfirsthead
  \hline
  \textbf{欄位名稱} & \textbf{欄位代號} & \textbf{定義} & \textbf{型態} \\
  \hline
  \endhead
  
  識別碼 & id & 資料的唯一識別碼 & String \\
  \hline
  名稱 & name & 地點的名稱 & String \\
  \hline
  描述 & description & 地點的詳細說明 & String \\
  \hline
  標籤 & tags & 地點相關的標籤,用於篩選和搜尋 & Set<String> \\
  \hline
  建立時間 & createdTime & 地點建立的時間 & DateTime \\
  \hline
  更新時間 & updatedTime & 地點最後一次被更新的時間 & DateTime \\
  \hline
  緯度 & lat & 地點的緯度 & Double \\
  \hline
  經度 & lon & 地點的經度 & Double \\
  \hline
  海拔 & ele & 地點的海拔 & Double \\
  \hline
  管理識別碼 & relatedId & 使用或管理該點的資源識別碼 & String \\
  \hline
\end{longtable}

\subsubsection{地點範例}

\begin{longtable}{|c|c|c|}
  \hline
  \textbf{欄位名稱} & \textbf{範例 1} & \textbf{範例 2} \\
  \hline
  \endfirsthead
  \hline
  \textbf{欄位名稱} & \textbf{範例 1} & \textbf{範例 2} \\
  \hline
  \endhead
  
  id & WPT001 & WPT002 \\
  \hline
  name & 臺北101 & 西門町 \\
  \hline
  description & 台灣最高的大樓 & 台北的購物和娛樂區 \\
  \hline
  tags & \{"地標", "商業"\} & \{"購物", "娛樂"\} \\
  \hline
  createdTime & 2023-10-10 12:34:56 & 2023-10-13 14:56:00 \\
  \hline
  updatedTime & 2023-10-11 15:30:00 & 2023-10-13 18:20:00 \\
  \hline
  lat & 25.0330 & 25.0418 \\
  \hline
  lon & 121.5654 & 121.5081 \\
  \hline
  ele & 508 & 10 \\
  \hline
  relatedId & TRACK001 & TRACK002 \\
  \hline
\end{longtable}
