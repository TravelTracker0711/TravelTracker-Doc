\noindent
\textbf{參賽隊名:} TraTracker \\
\textbf{作品名稱:} 旅遊紀錄整理工具 \\
\textbf{系統名稱:} TravelTracker

\section{系統目的與範圍}

TravelTracker 的主要目的是解決旅遊記錄、整理的不便,創造一個便捷的方式在旅程中輕鬆地記錄和整理旅遊資料。通過使用地圖、AI對話和自動標籤等技術,讓使用者能快速找到和整理旅遊途中的美好記錄。

\section{TravelTracker 提供的服務}
\begin{enumerate}
  \item \textbf{快速記錄軌跡與多媒體資料}:不用花時間開啟各種程式,能以最快的速度記錄回憶。
  \item \textbf{自動影像載入與地點推算}:自動從手機載入影像,並根據拍攝時間與軌跡資訊推算拍攝地點。
  \item \textbf{以拍攝地點分組旅遊資料}:利用地理資訊,讓使用者依照地點快速找到旅遊回憶。
  \item \textbf{AI辨識標記與對話整理}:使用AI技術自動標記圖像類別,並透過AI對話技術輕鬆整理旅遊資料。
  \item \textbf{全方位多媒體支援}:同時支援照片、影片、錄音、文字等資料形式,提供多元化的記錄方式。
\end{enumerate}

\section{TravelTracker 涵蓋的技術}
\begin{enumerate}
  \item \textbf{地圖軌跡整合技術}:使用地圖作為主要界面,結合GPS訊號和拍攝地點的推算,使旅遊資料更具關聯性和直觀性。
  \item \textbf{人工智慧辨識與對話技術}:利用AI分析照片內容進行自動標記,並結合自然語言處理技術,讓使用者能夠與AI進行對話,快速整理旅遊資料。
  \item \textbf{跨平台兼容性}:支援不同的移動設備,透過適應性設計讓使用者在不同設備上都能夠方便地使用。
\end{enumerate}
