\section{軟體或硬體模組設計}

\newcommand{\customfig}[2]{
  \begin{figure}[H]
    \centering
    \includegraphics[width=0.9\linewidth]{../assets/#1.png}
    \caption{#2}
    \label{#2}
  \end{figure}
}

\customfig{UML-Entity}{Entity 類別關係圖}

\begin{itemize}
  \item TravelDataConfig, TravelTrack, Trkseg, Asset 和 Wpt:這些實體描述了旅行相關的數據。其中,TravelDataConfig是描述資料的基礎配置;TravelTrack是完整的旅程,包含了多個Trkseg(軌跡)和Asset(旅遊資料);Trkseg是一段旅遊軌跡,包含了多個Wpt(路點)。
  \item AssetType:定義了不同的資產類型,如圖片、視頻、音頻、文字或是未知。
\end{itemize}

\customfig{UML-DataSource}{DataSource 類別關係圖}

\begin{itemize}
  \item DataSource 和 DataSourceItem:這部分定義了一個抽象的數據源,其中包含了基本的CRUD操作(創建、讀取、更新和刪除)。DataSourceItem是每個存儲在數據源中的項目的基礎,並有一個id。
  \item InMemoryDataSource:是DataSource的具體實現,用於在記憶體中存儲資料。它提供了兩個額外的方法,isMatch 和 postProcessRead,可以讓實作的類別覆寫匹配邏輯與後處理讀取結果。
  \item TravelTrackItem 和 TravelTrackItemCriteria:TravelTrackItem是具有名稱、描述、標籤和時間戳的特定項目。而TravelTrackItemCriteria是用於搜索特定條目的條件或過濾器。
\end{itemize}

\customfig{UML-Repository}{Repository 類別關係圖}

\begin{itemize}
  \item Repository 和 DataSource:Repository是一個抽象層,允許進行CRUD操作,而DataSource是其下一層的具體存儲實現。Repository是用於業務邏輯和數據存取的橋樑。
  \item TravelTrackRepository:是具體的Repository實現,用於TravelTrack的CRUD操作。它有多個InMemory DataSources來管理不同的項目(例如TravelTrackItem、TrksegItem等)。
  \item TravelTrackItemConverter:這是用於將TravelTrack和DataSource的Items類別互相轉換的工具。
\end{itemize}